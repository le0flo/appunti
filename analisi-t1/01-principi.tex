\documentclass{subfiles}

\usepackage{amsfonts}
\usepackage{amssymb}
\usepackage{amsmath}

\begin{document}

\section{Principi}

\subsection{Principio di induzione}

Il principio di induzione è un metodo generico per dimostrare un teorema qualsiasi con dominio in $\mathbb{N}$.
Il procedimento di prova per induzione prevede un affermazione o uguaglianza che dipende da $n$ e viene indicata con: $P(n)$.\\

\noindent
Data l'affermazione $P(n)$, il principio prova la precedente, al verificarsi delle seguenti condizioni:

\begin{enumerate}
    \item $P(0)$
    \item $P(k)$
\end{enumerate}

\noindent
Se queste condizioni si verificano, allora: $P(n)$ è vera $\forall n \in \mathbb{N}$.

\subsection{Principio del minimo}

Ogni sottoinsieme di $\mathbb{N}$ ha un minimo.

\subsection{Principio del valore intermedio}

Data una funzione continua $f(x)$ e due punti $f(a)$ e $f(b)$, allora:

$$
\exists c \in ] a, b [ : f(c) = \lambda
$$

\subsection{Piccole nozioni}

\begin{itemize}
    \item L'insieme dei numeri reali non è numerabile
    \item Radice aritmetica: $\sqrt{n} = m : m \in \mathbb{N}$
    \item Seno iperbolico: $\sinh(x) = \frac{e^x - e^{-x}}{2}$
    \item Coseno iperbolico: $\cosh(x) = \frac{e^x + e^{-x}}{2}$
    \item Tangente iperbolica: $\tanh(x) = \frac{\sinh(x)}{\cosh(x)}$
\end{itemize}

\subsection{Maggioranti e minoranti}

Dato $\mathbb{A} = \{ a \in \mathbb{R} \}$:

\begin{itemize}
    \item Maggiorante: $M \in \mathbb{R} \land \forall a \in \mathbb{A} : a \leq M$
    \item Minorante: $m \in \mathbb{R} \land \forall a \in \mathbb{A} : a \geq M$
\end{itemize}

\noindent
Se l'insieme $\mathbb{A}$ ammette un maggiorante si dice superiormente limitato, se ammette un minorante si dice inferiormente limitato e se ammette entrambe si dice limitato.

\begin{itemize}
    \item Il massimo di un insieme $\mathbb{A}$, se esiste, è il più piccolo dei maggioranti
    \item Il minimo di un insieme $\mathbb{A}$, se esiste, è il più grande dei minoranti
\end{itemize}

\subsection{Numeri complessi}

$$
x^2 + 1 = 0, \text{ impossibile in } \mathbb{R}
$$

\noindent
Si risolve introducendo $i$: $x^2 = -1 \to x = \sqrt{-1} \to i^2 = -1$.
Un numero complesso è composto da due parti:

\begin{itemize}
    \item parte reale: $a \in \mathbb{R}$
    \item parte complessa: $i * b : b \in \mathbb{R}$
\end{itemize}

\subsection{Notazione dei numeri complessi}

$$
z = a + ib
$$

\subsection{Teorema fondamentale dell'algebra}

\begin{description}
    \item[Tesi] $p(x) = a_n x^n + a_{n-1} x^{n-1} + \dots + a_1 x^1 + a_0$
    \item[Ipotesi] Il polinomio $p(x)$ ha $n$ radici in $\mathbb{C}$ tenendo in conto la moltiplicabilità: $a_0, \dots , a_n \in \mathbb{C}; a_n \neq 0$.
\end{description}

\end{document}
