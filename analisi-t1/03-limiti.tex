\documentclass{subfiles}

\usepackage{amsfonts}
\usepackage{amssymb}
\usepackage{amsmath}

\begin{document}

\section{Limiti}

\subsection{Definizione con le serie}

$$
\lim_{n \to +\infty} f(a_n) = L
$$

\subsection{Definizione con gli intorni}

La funzione $f(x)$ si avvicina arbitrariamente ad L, a patto che $x$ sia abbastanza vicino a $x_0$.

$$
\forall \varepsilon > 0, \exists \delta > 0 :
\forall n > \delta :
\forall x \in \space ] x_0 - \delta, x_0 + \delta [ \space \backslash \space \{x_0\}
\implies f(x) \in \space ] L - \varepsilon, L + \varepsilon [
$$

\begin{itemize}
    \item $\varepsilon$ = intorno al limite della funzione per quel valore
    \item $\delta$ = intorno al valore della funzione
    \item $L$ = il valore del limite
\end{itemize}

\noindent
Nel caso il limite $\in [-\infty, +\infty]$ allora usiamo la notazione $M > 0$ oppure $M < 0$ per indicare l'intorno.

\subsection{Gerarchia degli infiniti}

Seguono alcune funzioni in ordine crescente di chi permette a $n$ di tendere ad infinito più velocemente.

\begin{enumerate}
    \item $\ln(n)$
    \item $n^a : (a > 0)$
    \item $A^n : (a > 1)$
    \item $n!$
    \item $n^n$
\end{enumerate}

\subsection{Algebra dei limiti}

Dati i limiti $L, M, x_0 \in [ -\infty , +\infty ]$, $\lim_{x \to x_0} f(x) = L$, $\lim_{x \to x_0} g(x) = M$ allora:

\begin{itemize}
    \item $\lim_{x \to x_0} (f(x) + g(x)) = L + M$
    \item $\lim_{x \to x_0} (f(x) - g(x)) = L - M$
    \item $\lim_{x \to x_0} (f(x) * g(x)) = L * M$
\end{itemize}

\subsection{Limiti notevoli}

\begin{itemize}
    \item $\lim_{x \to 0} \frac{\sin(x)}{x} = 1$
    \item $\lim_{x \to 0} \frac{\cos(x) - 1}{x^2} = -\frac{1}{2}$
    \item $\lim_{x \to 0} \frac{e^x - 1}{x} = 1$
    \item $\lim_{x \to 0} \frac{\ln(1 + x)}{x} = 1$
\end{itemize}

\end{document}
