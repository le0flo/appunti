\documentclass{subfiles}

\usepackage{amsfonts}
\usepackage{amssymb}
\usepackage{amsmath}

\begin{document}

\section{Reti sequenziali sincrone}

\subsection{Vantaggi e limiti di una RSA}

Il vantaggio principale è che la rete risponde non appena cambiano gli ingressi, grazie ai segnali in retroazione diretta.
Avere i segnali in retroazione diretta, per una rete, porta ad una serie di malfunzionamenti dovuti principalmente ai ritardi con cui si aggiornano i segnali in retroazione.
La difficoltà di realizzare un progetto cresce molto al crescere della complessità della funzione da realizzare.
In pratica le RSA vengono usate soltanto per realizzare memorie, latch e flip-flop.

\subsection{Reti sequenziali sincrone}

Mentre in una rete sequenziale asincrona, l'aggiornamento dello stato avviene immediatamente, anche durante il transitorio, nelle reti sequenziali sincrone, l'aggiornamento dello stato avviene quando tutti i segnali di stato sono a regime.
Questo permette alle RSS di sfruttare i seguenti vantaggi:

\begin{itemize}
    \item Espressioni minime
    \item Codice d'ingresso arbitrario
    \item Codice di stato arbitrario
    \item Progettazione più semplice rispetto alle RSA
\end{itemize}

\subsection{Campionamento periodico}

Per leggere lo stato futuro ed aggiornare lo stato presente a regime, bisogna utilizzare la tecnica del campionamento periodico.
Campionare, o memorizzare, è l'azione che ci permette di leggere lo stato futuro ed aggiornare lo stato presente, ed è una azione che va eseguita periodicamente, perciò ad intervalli regolari, sufficientemente lunghi.
Grazie al campionamento, lo stato presente si aggiorna ogni $T_0$ unità di tempo, con il valore che lo stato futuro ha assunto alla fine del periodo, filtrando tutte le oscillazioni e le corse.
Per campionare con periodo $T_0$, si possono utilizzare $k$ flip-flop D (tanti quanti i bit di stato da memorizzare), il cui ingresso $CLK$ è collegato ad un segnale periodico di clock, con periodo $T_0$, che posso generare da un circuito oscillatore, con frequenza $\frac{1}{T_0}$.

\subsection{Struttura generale di una RSS}

Ogni rete combinatoria con anelli in retroazione su cui sono inseriti FF-D comandati da un generatore di clock in comune è quindi una RSS.
La rete ha $k$ bit di stato se ci sono $k$ segnali in retroazione.
I FF-D agiscono da memoria dello stato presente e ritardo costante nel suo aggiornamento.
Stato futuro e stato presente sono entrambi osservabili.
Sequenza tipica di lavoro di una rete sequenziale sincrona:

\begin{enumerate}
    \item Al fronte di salita del segnale di clock, vengono campionati i bit di stato futuro dai FF-D
    \item Dopo un tempo pari al tempo di risposta dei FF-D, lo stato futuro diventa stato presente
    \item L'eventuale cambiamento dello stato presente e l'eventuale cambiamento degli ingressi porta la rete a calcolare una nuova uscita e un nuovo stato futuro (gli stati spuri intermedi, se esistenti, vengono ignorati dalla rete)
    \item I nuovi bit di stato futuro sono a regime con un anticipo sulla fine del periodo $T_0$ pari almeno al tempo di set-up dei FF-D, pronti per essere campionati alla fine del periodo e iniziare un nuovo ciclo
\end{enumerate}

\subsection{Temporizzazione di una RSS}

Le RSS non funzionano ad inseguimento degli ingressi come le reti combinatorie o le RSA.
Ogni stato presente è stabile per almeno un periodo di clock.
Lo stato futuro e l'uscita possono modificarsi anche se non cambiano gli ingressi.
Ciò che fa evolvere stato ed uscita di una RSS è l'arrivo di un fronte di salita del clock e nient'altro.\\

\noindent
L'unico vincolo in una RSS è che il periodo di clock $T_0$ ha un valore minimo.
I nuovi ingressi e lo stato presente devono avere tempo di propagarsi attraverso la funzione $G$.
Oltre al ritardo della rete $G$, anche gli FF-D impongono dei vincoli su $T_0$:

\begin{itemize}
    \item I bit di stato futuro $Y_{0 \dots k-1}$ devono rimanere costanti sia durante $t_{su}$, sia durante $t_h$
    \item I bit di stato presente $y_{0 \dots k-1}$ saranno ingressi stabili per $G$ dopo $t_r$ dal fronte del clock
\end{itemize}

\noindent
Per garantire il corretto funzionamento della rete, il periodo $T_0$ deve essere maggiore della somma di:

\begin{itemize}
\item tempo necessario per avere ingressi a regime
\item ritardo della funzione $G$
\item tempo di set-up dei FF-D
\end{itemize}

\subsection{Finite State Machine}

Le RSS sono un caso particolare di automa a stati finiti $M$, ovvero un sistema matematico:

$$
FSM = \{ I,U,S,F,G \}
$$

\begin{description}
    \item[$I$] alfabeto di ingresso
    \item[$U$] alfabeto di uscita
    \item[$S$] insieme degli stati
\end{description}

\begin{description}
    \item[$F: S \times I \to U$] funzione di uscita
    \item[$G: S \times I \to S$] funzione di aggiornamento dello stato interno
\end{description}

\noindent
In questo caso la memoria che mantiene il vecchio stato $s$ fino a quando non è necessario sostituirlo con il nuovo stato $s^*$ è realizzata dai FF-D.
È possibile realizzare RSS in accordo al modello di Mealy o al modello di Moore e la scelta ha impatti più significativi di quanti non ne avesse per le RSA.

\subsection{Sincronizzazione di ingressi asincroni}

Il modello delle RSS assume ingressi sincroni, ovvero che variano una sola volta in ogni ciclo di clock al fronte del clock stesso.
Non è possibile rendere sincrono un segnale che varia con una frequenza maggiore del clock stesso, perciò in questo caso bisogna aumentare la frequenza di clock.
Se invece la frequenza con cui varia l'ingresso è minore del di quella del clock, allora la sincronizzazione è possibile e il risultato è il segnale $X_{sync}$, le cui variazioni rispecchiano quelle di $X$ (il segnale asincrono), ma avvengono in corrispondenza dei fronti di salita del clock.\\

\noindent
Una rete che realizza un segnale $X_{sync}$ partendo da un ingresso asincrono $X$ viene detta sincronizzatore.
Il più semplice tra i sincronizzatori è un FF-D.
Per diminuire la probabilità che la rete veda un segnale in metastabilità, il sincronizzatore viene spesso realizzato con 2 o più FF-D in cascata.

\subsection{Segnali $A\_$}

Un segnale di ingresso asincrono viene indicato dal suo nome preceduto da $A\_$.\\

\noindent
esempio:

\begin{itemize}
    \item $A\_RESET$ e $A\_RES$ $\implies$ Segnali di reset asincroni
    \item $RESET$ e $RES$ $\implies$ Segnali di reset sincroni
\end{itemize}

\subsection{Sintesi di una RSS}

\subsubsection{Sintesi formale}

Il procedimento di sintesi formale di una rete sequenziale sincrona è formato da 5 passi e consente di dedurre lo schema logico dalle specifiche di comportamento:

\begin{enumerate}
    \item Comprensione delle specifiche e individuazione del grafo degli stati
    \item Definizione della tabella di flusso a partire dal grafo degli stati
    \item Codifica degli stati e definizione della tabella delle transizioni
    \item Sintesi della parte combinatoria
    \item Disegno dello schema logico
\end{enumerate}

\subsubsection{Sintesi diretta}

La progettazione diretta di una RSS si esegue combinando RSS notevoli più semplici (registri, shift register, contatori, $\dots$) tramite opportuna logica combinatoria.

\end{document}
