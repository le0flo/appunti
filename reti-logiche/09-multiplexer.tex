\documentclass{article}

\usepackage{amsfonts}
\usepackage{amssymb}
\usepackage{amsmath}

\begin{document}

\section{Multiplexer}

esempio:\\

\noindent
Un multiplexer o selettore a 2 vie.
Un selettore è l'equivalente, in hardware, di un "if".
Permette di decidere tramite un segnale $A$, quale tra le due vie d'ingresso, $I_0$ o $I_1$, sarà replicata dall'uscita.
La tabella della verità e l'equazione SP risultante sono le seguenti:

\begin{center}
\begin{tabular}{ |c|c|c|c|c| }
\hline
$A \backslash I_1,I_0$ & 00 & 01 & 11 & 10 \\
\hline
\hline
0 & 0 & 1 & 1 & 0 \\
1 & 0 & 0 & 1 & 1 \\
\hline
\end{tabular}
\end{center}

$$
Z = A'I_0 + A I_1
$$

\subsection{Multiplexer (MUX) generico}

Il concetto di selettore può essere generalizzato a $n$ bit di indirizzo.
Gli $n$ bit di indirizzo selezionano uno tra i $2^n$ ingressi detti "vie" o anche "bit di programmazione", a seconda dell'uso che si fa del MUX.
Al crescere d $n$ cresce esponenzialmente il numero delle vie.
L'ingresso replicato sull'uscita si determina nel seguente modo:

$$
i = A_{n-1} 2^{n-1} + \dots + A_1 2^1 + A_0 2^0
$$

\subsection{Teorema dell'espansione di Shannon}

Data una funzione $F$ di $n$ variabili binarie, vale la relazione:

$$
F(x_1, \dots, x_i, \dots, x_n) = x_i F(x_1, \dots, 1, \dots, x_n) + \overline{x_i} F(x_1, \dots, 0, \dots, x_n)
$$

\noindent
La relazione duale è ugualmente valida:

$$
F(x_1, \dots, x_i, \dots, x_n) = (x_i + F(x_1, \dots, 0, \dots, x_n))(\overline{x_i} + F(x_1, \dots, 1, \dots, x_n))
$$

\subsection{Espressioni generali}

In generale, applicando ripetutamente il teorema di espansione di Shannon, è possibile dedurre le seguenti espressioni generali:

\begin{description}
	\item[Caso SP] ogni funzione di $n$ variabili è descritta da una espressione in cui compaiono, in somma logica, tutti i mintermini di $n$ variabili, ciascuno in prodotto logico con il valore della funzione quando in ingresso compare la configurazione riconosciuta dal mintermine

	$$
    F(x_1, x_2, \dots, x_n) = \sum_{i=0}^{2^n-1} m(i) * F(i)
    $$

	\item[Caso PS] ogni funzione di $n$ variabili è descritta da una espressione in cui compaiono, in prodotto logico, tutti i maxtermini di $n$ variabili, ciascuno in somma logica con il valore della funzione quando in ingresso compare la configurazione riconosciuta dal maxtermine

	$$
    F(x_1, x_2, \dots, x_n) = \prod_{i=0}^{2^n-1} (M(i) + F(i))
    $$
\end{description}

\subsection{Il MUX come rete programmabile}

Il MUX si adatta bene a realizzare l'espressione generale nel caso SP.
Con $n$ variabili occorre un MUX a $n$ bit di indirizzo.
Il MUX viene utilizzato, in questo caso, come generatore di funzioni.

\subsection{MUX in forma integrata}

Esistono MUX a 2,4,8 o 16 bit di programmazione.
Il circuito integrato ha un numero di pin limitati, ma gli ingressi di un MUX crescono esponenzialmente, perciò saranno disponibili in forma integrata, al massimo MUX a 16 bit di programmazione.

\end{document}
