\documentclass{article}

\usepackage{amsfonts}
\usepackage{amssymb}
\usepackage{amsmath}
\usepackage{multicol}

\begin{document}

\section{Sintesi RSA}

5 passaggi:

\begin{enumerate}
    \item Individuazione del Grafo degli stati
    \item Definizione della Tabella di flusso
    \item Codifica degli stati e definizione della Tabella delle transizioni
    \item Sintesi della rete combinatoria di uscita e stato futuro
    \item Schema logico che includa anche ingresso di reset, se presente
\end{enumerate}

\subsection{Tabella di Flusso}

2 controlli:

\begin{itemize}
    \item In ogni riga ci deve essere almeno una condizione di stabilità
    \item Le situazioni di instabilità devono indicare uno stato futuro stabile nella colonna
\end{itemize}

\noindent
esempio:

\begin{center}
\begin{tabular}{ |c|c|c| }
\hline
& 0 & 1 \\
\hline
\hline
A & A,0 & B,0 \\
B & C,- & B,0 \\
C & C,1 & D,1 \\
D & A,- & D,1 \\
\hline
\end{tabular}
\end{center}

\subsection{Tabella delle transizioni}

In ogni rete sequenziale, lo stato è rappresentato da una configurazione binaria dei bit di stato.
Bisogna quindi scegliere un modo per codificare gli stati.
Data la codifica, posso tradurre la tabella di flusso in tabella delle transizioni, sostituendo ad ogni stato la codifica binaria.
Non tutte le codifiche producono un funzionamento corretto della rete.\\

\noindent
esempio:

\begin{center}
\begin{tabular}{ |c|c|c| }
\hline
& 0 & 1 \\
\hline
\hline
A & A,0 & B,0 \\
B & C,- & B,0 \\
C & C,1 & D,1 \\
D & A,- & D,1 \\
\hline
\end{tabular}

$$\Downarrow$$

\begin{tabular}{ |c|c|c| }
\hline
& 0 & 1 \\
\hline
\hline
0 & A & B \\
1 & D & C \\
\hline
\end{tabular}

$$\Downarrow$$

\begin{tabular}{ |c|c|c| }
\hline
& 0 & 1 \\
\hline
\hline
A=00 & 00,0 & 01,0 \\
B=01 & 11,- & 01,0 \\
C=11 & 11,1 & 10,1 \\
D=10 & 00,- & 10,1 \\
\hline
\end{tabular}
\end{center}

\subsection{Espressioni combinatorie}

La tabella delle transizioni è in realtà una composizione di TdV combinatorie, ovvero di mappe di Karnaugh,
con cui è possibile eseguire la sintesi combinatoria delle funzioni che implementano il comportamento richiesto con le metodologie già studiate.\\

\noindent
esempio:

\begin{center}
\begin{tabular}{ |c|c|c| }
\hline
& 0 & 1 \\
\hline
\hline
A=00 & 00,0 & 01,0 \\
B=01 & 11,- & 01,0 \\
C=11 & 11,1 & 10,1 \\
D=10 & 00,- & 10,1 \\
\hline
\end{tabular}

$$\Downarrow$$

\begin{multicols}{2}
\begin{tabular}{ |c|c|c|c|c| }
\hline
& 00 & 01 & 11 & 10 \\
\hline
\hline
0 & 0 & - & 1 & - \\
1 & 0 & 0 & 1 & 1 \\
\hline
\end{tabular}

$Z = y1$
\end{multicols}

\begin{multicols}{2}
\begin{tabular}{ |c|c|c|c|c| }
\hline
& 00 & 01 & 11 & 10 \\
\hline
\hline
0 & 0 & 1 & 1 & 0 \\
1 & 0 & 0 & 1 & 1 \\
\hline
\end{tabular}

$Y_1 = xy_1 + x'y_0 + y_1y_0$
\end{multicols}

\begin{multicols}{2}
\begin{tabular}{ |c|c|c|c|c| }
\hline
& 00 & 01 & 11 & 10 \\
\hline
\hline
0 & 0 & 1 & 1 & 0 \\
1 & 1 & 1 & 0 & 0 \\
\hline
\end{tabular}

$Y_0 = xy_1' + x'y_0 + y_1'y_0$
\end{multicols}
\end{center}

\end{document}
