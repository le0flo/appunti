\documentclass{article}

\usepackage{amsfonts}
\usepackage{amssymb}
\usepackage{amsmath}

\begin{document}

\section{Reti di costo minimo}

\subsection{Ritardi e velocità}

Quando cambia un ingresso di un gate, l’uscita non cambia istantaneamente, ma dopo un tempo $\tau_p$ che dipende dalla tecnologia utilizzata.
Questo ritardo varia da gate a gate e anche se il passaggio è da H a L o viceversa.
Nel caso peggiore, il ritardo totale della rete è dato dalla somma dei ritardi dei gate sul percorso più lungo tra ingressi e uscite.
Si assegna il ritardo peggiore alla rete complessiva.

\subsection{Complessità e velocità}

Per confrontare complessità e velocità di risposta di reti combinatorie equivalenti, si usano i seguenti indicatori:

\begin{description}
    \item[$N_{gate}$] il numero di gate nella rete (maggiore è l'$N_{gate}$ , maggiore è la complessità)
    \item[$N_{conn}$] il numero di connessioni in una rete (maggiore è l'$N_{conn}$ , maggiore è la complessità)
    \item[$N_{casc}$] il numero massimo di gate disposti in cascata, ovvero in serie tra ingressi e uscite (minore è l'$N_{gate}$ , maggiore è la velocità)
\end{description}

\subsection{Rete di "costo minimo"}

Ipotesi:

\begin{itemize}
    \item Ingressi disponibili in forma vera e negata
    \item fan-in dei gate quando serve
\end{itemize}

\noindent
Una rete combinatoria, per essere considerata "di costo minimo", è una rete con:

\begin{itemize}
    \item Non più di 2 gate in cascata tra ingressi e uscita
    \item Minimo numero di gate
    \item Minimo numero di ingressi per gate
\end{itemize}

\noindent
Il numero di gate e/o connessioni della rete di costo minimo di tipo SP è in generale diverso da quello della rete di costo minimo di tipo PS.\\

\noindent
E' possibile che più espressioni dello stesso tipo (SP o PS) siano minime (abbiano cioè valori uguali di $N_{gate}, N_{conn}$ e $N_{casc} \leq 2$).

\subsection{Implicanti e implicanti primi}

Viene detto implicante, un termine di $n$ ingressi che assume il valore 1 solo la dove la funzione vale 1 o per indifferenza.
Un implicante che cessa di essere tale quando si rimuove un suo letterale viene detto implicante primo.
Un implicante primo essenziale è l'unico ad assumere valore 1 per alcune configurazioni degli ingressi in cui la funzione assume valore 1 (non per indifferenza).
L'espressione minima SP è la somma di implicanti primi essenziali.\\

\noindent
esempio:

\begin{center}
\begin{tabular}{ |c|c|c|c| }
\hline
$a$ & $b$ & $c$ & $z$ \\
\hline
\hline
0 & 0 & 0 & 1 \\
0 & 0 & 1 & 1 \\
0 & 1 & 0 & 1 \\
0 & 1 & 1 & - \\
1 & 0 & 0 & - \\
1 & 0 & 1 & 0 \\
1 & 1 & 0 & 0 \\
1 & 1 & 1 & 1 \\
\hline
\end{tabular}
\end{center}

\noindent
Implicanti rispetto alla TdV:

\begin{itemize}
    \item 3 termini (mintermini): $a'b'c', a'b'c, a'bc', a'bc, ab'c', abc$
    \item 2 termini: $a'b', a'b, a'c, a'c', b'c', bc$
    \item 1 termine: $a'$
\end{itemize}

\begin{center}
\begin{tabular}{ |c|c|c|c|c| }
\hline
$a$ & $b$ & $c$ & $z$ & implicanti primi attivi \\
\hline
\hline
0 & 0 & 0 & 1 & $a', b'c'$ \\
0 & 0 & 1 & 1 & $a'$ \\
0 & 1 & 0 & 1 & $a'$ \\
0 & 1 & 1 & - & $a', bc$ \\
1 & 0 & 0 & - & $b'c'$ \\
1 & 0 & 1 & 0 & \\
1 & 1 & 0 & 0 & \\
1 & 1 & 1 & 1 & $bc$ \\
\hline
\end{tabular}
\end{center}

\noindent
Implicanti primi: $a', bc$

$$
F(a,b,c) = a' + bc
$$

\subsection{Implicati e implicati primi}

Viene detto implicato, un termine di $n$ ingressi che assume il valore 0 solo la dove la funzione vale 0 o per indifferenza.
Un implicato che cessa di essere tale quando si rimuove un suo letterale viene detto implicato primo.
Un implicato primo essenziale è l'unico ad assumere valore 0 per alcune configurazioni degli ingressi in cui la funzione assume valore 0 (non per indifferenza).
L'espressione minima PS è il prodotto degli implicati primi essenziali.\\

\noindent
esempio:

\begin{center}
\begin{tabular}{ |c|c|c|c| }
\hline
$a$ & $b$ & $c$ & $z$ \\
\hline
\hline
0 & 0 & 0 & 1 \\
0 & 0 & 1 & 1 \\
0 & 1 & 0 & 1 \\
0 & 1 & 1 & - \\
1 & 0 & 0 & - \\
1 & 0 & 1 & 0 \\
1 & 1 & 0 & 0 \\
1 & 1 & 1 & 1 \\
\hline
\end{tabular}
\end{center}

\begin{description}
    \item[Implicati] $a+b'+c', a'+b+c, a'+b+c', a'+b'+c, a'+c, a'+b$
    \item[Implicati primi] $a'+c, a'+b, a+b'+c'$
    \item[Implicati primi essenziali] $a'+c, a'+b$
\end{description}

$$
F(a,b,c) = (a'+c)(a'+b)
$$

\end{document}
