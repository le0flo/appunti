\documentclass{subfiles}

\usepackage{amsfonts}
\usepackage{amssymb}
\usepackage{amsmath}

\begin{document}

\section{Informazione}

\subsection{Codifica binaria}

In una macchina digitale, le informazioni vengono rappresentate sotto forma di bit (0 e 1).
Questo sistema è detto sistema binario.
Ci sono diversi modi per rappresentare le informazioni. Prendendo come esempio il numero 28:

\begin{center}
\begin{tabular}{ |c|c|c| }
\hline
Sistema binario & Sistema decimale & Sistema esadecimale \\
\hline
\hline
11100 & 28 & 1C \\
\hline
\end{tabular}
\end{center}

\noindent
Viene detto numero in base $n$ il numero che viene rappresentando usando l'$n$-esimo sistema numerico.

\subsection{Cambio di base}

Per convertire un numero da un sistema all'altro, esistono diversi metodi, dette conversioni. Alcune di queste sono:

\subsubsection{Conversione iterativa}

In base $n$: si divide un numero in parte intera e parte decimale.

\begin{description}
    \item[Parte intera] si divide per $n$ il numero iniziale, ottenendo un quoziente intero e un resto (*0 o 1*); si segna il resto e si ripete l'operazione dividendo il quoziente intero precedente, finché esso risulterà $= 0$; per finire, riscrivere tutti i resti "al contrario" (*partendo dall'ultimo calcolato e arrivando al primo*).
    \item[Parte decimale] si prende il numero iniziale e lo si moltiplica per $n$, segnando separatamente la parte intera e decimale del risultato. Poi bisogna ripetere lo stesso procedimento usando la parte decimale segnata precedentemente come fattore, iterando finché la parte decimale risulterà $= 0$ oppure in base alla precisione decisa.
\end{description}

\subsection{Numeri binari}

Per rappresentare un numero senza segno $N$ su una macchina digitale, usiamo il sistema binario, base $2$.
La quantità di numeri che possiamo rappresentare è determinata dalla quantità di bit che vengono assegnati.
Per un numero $n$ finito di bit, la quantità di numeri senza segno $N$ in base $2$ è:

$$
2^n - 1
$$

\subsection{Problemi con la conversione A/D}

Nel mondo reale, l'allineamento meccanico perfetto è impossibile, ovvero due segnali non possono variare contemporaneamente.
Per evitare letture scorrette nei punti di disallineamento, è opportuno utilizzare configurazioni relative a posizioni consecutive che differiscono di un solo valore.

\subsection{Codice di Gray}

A differenza del codice binario, il codice di Gray è una configurazione di bit che codifica informazioni adiacenti tali che differiscono soltanto di 1 bit.
Viene usato generalmente per ridurre errori nella codifica da analogico a digitale.

\subsection{Transcodifica}

Possiamo dividere i codici in esterno e interno.

\begin{description}
	\item[Esterno] ridondante e standardizzato
	\item[Interno] non ridondante
\end{description}

\subsection{Codici standard e proprietari}

\begin{description}
    \item[Standard] Un codice scelto da norme internazionali o da un costruttore molto diffuso, e permette a tutti di adottare quel codice per rendere le proprie informazioni facilmente interpretabili
    \item[Proprietario] Un codice scelto dal costruttore con l'unico scopo di interconnettere i propri prodotti.
\end{description}

\end{document}
