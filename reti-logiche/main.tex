\documentclass{article}

\usepackage{geometry}
\usepackage{subfiles}
\usepackage{amsfonts}
\usepackage{amssymb}
\usepackage{amsmath}
\usepackage{multicol}

\title{Reti Logiche}
\geometry{margin=1in}

\begin{document}

\maketitle

\subfile{reti-logiche/01-macchine-digitali}
\subfile{reti-logiche/02-informazione}
\subfile{reti-logiche/03-reti-logiche}
\subfile{reti-logiche/04-reti-combinatorie}
\subfile{reti-logiche/05-reti-di-costo-minimo}
\subfile{reti-logiche/06-mappe-di-karnaugh}
\subfile{reti-logiche/07-algebre-binarie}
\subfile{reti-logiche/08-decoder}
\subfile{reti-logiche/09-multiplexer}
\subfile{reti-logiche/10-reti-programmabili}
\subfile{reti-logiche/11-aritmetica-binaria}
\subfile{reti-logiche/12-reti-sequenziali-asincrone}
\subfile{reti-logiche/13-alee}
\subfile{reti-logiche/14-funzionamento-corretto-di-rsa}
\subfile{reti-logiche/15-sintesi-rsa}
\subfile{reti-logiche/16-memorie-binarie}
\subfile{reti-logiche/17-reti-sequenziali-sincrone}
\subfile{reti-logiche/18-registro}
\subfile{reti-logiche/19-monoimpulsore}
\subfile{reti-logiche/20-contatore}
\subfile{reti-logiche/21-riconoscitore-di-sequenze}
\subfile{reti-logiche/22-clock-gating-e-skew}

\end{document}
