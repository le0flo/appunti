\documentclass{subfiles}

\usepackage{amsfonts}
\usepackage{amssymb}
\usepackage{amsmath}

\begin{document}

\section{Memorie binarie}

Uno degli usi più importanti delle RSA è la realizzazione di memorie binarie, ovvero circuiti che hanno lo scopo di memorizzare il valore di un bit.
Sono categorizzati come componenti primitivi, inoltre sono i principali blocchi di costruzione per le reti sincrone.
Ci sono 3 tipi principali di memorie:

\begin{itemize}
    \item Latch SR
    \item Latch CD
    \item Flip-flop D
\end{itemize}

\noindent
Si differenziano per quale sequenza di ingressi porta alla scrittura di un bit, quindi quando sono sensibili a comandi di modifica dell’informazione memorizzata.

\subsection{Latch SR}

La più semplice memoria binaria.
Questo componente ha:

\begin{description}
    \item[2 Ingressi] set $S$ e reset $R$
    \item[2 Uscite] $Q$ e il suo complemento $Q'$
\end{description}

\noindent
Il comportamento della rete è il seguente:

\begin{center}
\begin{tabular}{ |c|c|c|c| }
\hline
Comando & Ingresso $S$ & Ingresso $R$ & Uscita $Q$  \\
\hline
\hline
Memorizza & 0 & 0 & Ultimo valore scritto \\
Scrivi $1$ & 1 & 0 & 1 \\
Scrivi $0$ & 0 & 1 & 0 \\
\hline
\end{tabular}
\end{center}

\noindent
La coppia di ingressi 11 non può presentarsi per un corretto funzionamento della rete.

\subsubsection{Sintesi del latch SR}

\begin{center}
\begin{multicols}{2}
\begin{tabular}{ |c|c|c|c|c|c| }
\hline
& 00 & 01 & 11 & 10 & $Q$ \\
\hline
\hline
A & A & A & - & B & 0 \\
B & B & A & - & B & 1 \\
\hline
\end{tabular}

Tabella di flusso
\end{multicols}

\begin{multicols}{2}
\begin{tabular}{ |c|c|c|c|c|c| }
\hline
& 00 & 01 & 11 & 10 & $Q$ \\
\hline
\hline
A=0 & 0 & 0 & - & 1 & 0 \\
B=1 & 1 & 0 & - & 1 & 1 \\
\hline
\end{tabular}

Tabella delle transizioni
\end{multicols}

\begin{multicols}{2}
\begin{description}
\item[Espressione SP] $Y = S + yR'$
\item[Espressione PS] $Y = R' * (y+S)$
\end{description}

Sintesi combinatoria
\end{multicols}
\end{center}

\subsubsection{Stato iniziale di un latch SR}

Come ogni RSA, il bit di stato di un latch SR avrà un valore casuale all'avvio della macchina.
Ci sono casi in cui, però, serve che il latch SR abbia uno stato iniziale predeterminato all'accensione, perciò esistono delle varianti del latch SR dotate di 2 ingressi aggiuntivi, normalmente attivi bassi, prioritari rispetto a $S$ e $R$:

\begin{description}
    \item[Preset $PRE'$] quando $PRE'=0$, il latch SR memorizza $1$ indipendentemente dai valori di $S$ e $R$
    \item[Clear $CLR'$] quando $CLR'=0$, il latch SR memorizza $0$ indipendentemente dai valori di $S$ e $R$
\end{description}

Anche per gli ingressi di preset e clear vale la regola che non possono mai essere contemporaneamente attivi.
Poiché gli ingressi di inizializzazione $PRE'$ e $CLR'$ non devono essere usati durante la normale operatività del latch ma solo all'avvio, sono rappresentati tipicamente in alto e in basso, rispetto agli altri ingressi, situati sul lato sinistro della rete.

\subsection{Metastabilità}

Uno stato stabile è uno stato in cui $Q=q$ (stato futuro uguale allo stato presente).
Idealmente, il bit di stato può assumere soltanto i valori H o L.
In realtà i segnali che si trovano sull'anello di retroazione, come tutti i segnali di una rete reale, sono segnali analogici.
Questo significa che i segnali possono assumere un valore anche leggermente superiore o inferiore ad H e L.\\

\noindent
Per garantire un corretto funzionamento di un circuito, esso deve rimanere stabile anche a fronte di disturbi sui segnali analogici sottostanti.
Se il segnale analogico da cui è derivato $q$ si sposta di poco all'interno della banda in cui viene interpretato, la risposta $Q$ della rete combinatoria deve riavvicinarsi al valore predefinito.
Se si allontanasse, amplificando il disturbo, la rete non riuscirebbe a garantire la stabilità in una situazione reale.
Questo si ottiene a livello elettronico, facendo si che la relazione ingresso-uscita analogica del circuito combinatorio $f_{rc}$, che poniamo in retroazione, abbia tratti a pendenza minore di 1 intorno ad H e L.\\

\noindent
Per unire due tratti di pendenza < 1, è inevitabile creare un tratto di pendenza > 1 che interseca di nuovo la retta $Q = q$.
Esiste quindi un'altro punto M oltre ai valori H e L in cui la rete può fermarsi.
Questo punto viene detto stato meta-stabile e se il una rete combinatoria si trova in tale punto, si dice che è in metastabilità.
Il problema è che non si può sapere in quale dei due stati si troverà la rete a seguito del disturbo, poiché dipende soltanto dal disturbo stesso, che è casuale.\\

\noindent
Un latch, per esempio, può trovarsi in uno stato meta-stabile se la durata degli ingressi non rispetta il vincolo $t_w$.

\subsection{Latch CD}

Il latch CD è un'altra memoria binaria.
La differenza principale da un latch SR è il modo in cui viene comandato.
Il latch CD memorizza il valore del segnale $D$ soltanto quando $C$ ha valore $1$.
Chi lo comanda non dice esplicitamente cosa scrivere come nel caso del latch SR, ma quando scrivere $D$, pilotando opportunamente $C$.
Quindi diventa possibile, pilotando correttamente il segnale $C$, condiviso da $n$ latch, memorizzare contemporaneamente $n$ dati binari.

\begin{center}
\begin{tabular}{ |c|c|c|c| }
\hline
Comando & Ingresso $C$ & Ingresso $D$ & Uscita $Q$  \\
\hline
\hline
Memorizza & 0 & - & Ultimo valore scritto \\
Scrivi $1$ & 1 & 1 & 1 \\
Scrivi $0$ & 1 & 0 & 0 \\
\hline
\end{tabular}
\end{center}

\noindent
Il campionamento del segnale $D$ avviene soltanto quando $C$ è alto ($1$).
Significa che il latch CD è una memoria level-triggered.
Il latch CD è utile perché la forma d'onda nell'ingresso $D$, pilotando opportunamente $C$, viene ripulita da andamenti indesiderati in transitorio o valori che non doveva essere memorizzati, quando la si va a leggere dall'uscita $Q$.
Per questo motivo è il componente di memoria usato all'interno del registro buffer e della memoria RAM.

\subsubsection{Sintesi formale}

$$
\begin{matrix}
Y=CD+C'y+Dy\\
Q=y
\end{matrix}
$$

\subsubsection{Sintesi diretta}

Si può realizzare una sintesi di un latch CD utilizzando i componenti precedentemente introdotti.
Avendo il latch SR, una RSA che può memorizzare un bit, basterebbe ideare una rete che transcodifichi il codice CD in codice SR.
La rete è costituita da 2 AND, corrispondenti ai due mintermini della tabella seguente:

\begin{center}
\begin{tabular}{ |c|c|c|c|c|c| }
\hline
Comando & $C$ & $D$ & $Q$ & $S$ & $R$ \\
\hline
\hline
Memorizza & 0 & - & Ultimo valore scritto & 0 & 0 \\
Scrivi $1$ & 1 & 1 & 1 & 1 & 0 \\
Scrivi $0$ & 1 & 0 & 0 & 0 & 1 \\
\hline
\end{tabular}
\end{center}

\begin{itemize}
    \item $S = CD$
    \item $R = CD'$
\end{itemize}

\subsubsection{Tempi di set-up, di hold e di risposta}

Avendo più gate in cascata, il transitorio del latch CD dura più di quello del latch SR.
Come nel latch SR, anche il comando di campionamento ($C=1$) deve avere una durata minima di $t_w$.
Non rispettare questo vincolo può causare metastabilità, come nel latch SR.
Esistono altri due tempi da considerare nel latch CD, rispettivamente prima e dopo il fronte di discesa di $C$.

\begin{description}
    \item[Tempo di set-up] il tempo di propagazione attraverso i gate
    \item[Tempo di hold] il tempo necessario per innescare la retroazione
\end{description}

\noindent
L'ingresso $D$ deve essere costante durante $t_{su}+t_h$ per garantire il corretto funzionamento del latch.
Questi vincoli evitano che la rete veda cambiamenti "simultanei" degli ingressi, proibiti nelle RSA.
L'uscita in forma vera e complementata è disponibile a partire dal fronte di salita di $C$, dopo un tempo di risposta $t_r$, tipicamente diverso se $Q$ passa da L a H o da H a L.
Se quando il segnale $C=1$, $D$ modifica il suo valore e lo mantiene per almeno il tempo di setup, allora la stessa modifica si riscontra anche su $Q$, con un ritardo pari al tempo di risposta (uscita trasparente).

\subsection{Flip-flop D}

Una RSA con due ingressi $CLK$ e $D$.
L'ingresso $CLK$ (clock) è tipicamente rappresentato da un triangolo.
$CLK$ svolge il ruolo di segnale di campionamento, $D$ il segnale campionato.
Il campionamento ha luogo quando $CLK$ transita dal valore logico 0 al valore logico 1 (campionamento su fronti di salita, positive edge-triggered).
$Q$, dopo un piccolo tempo dal fronte di salita, riflette l'ultimo valore campionato.\\

\noindent
Questo ritardo è fondamentale, in quanto garantisce che quando $Q$ cambia, la rete non è più in trasparenza.
Con il flip-flop D è possibile qualunque montaggio in retroazione, senza alcun rischio di instabilità.
Esistono due tipi principali di flip-flop D:

\begin{description}
    \item[Master-slave] realizzato con due latch CD in cascata con $C$ in forma vera e negata
    \item[Edge-triggered] 6 NAND invece che 8 e altri miglioramenti che vengono da scelte di progetto non intuitive, ottenute utilizzando strumenti di CAD (Computer-Aided Design)
\end{description}

\subsubsection{Tempi di set-up, di hold e di risposta}

I tempi caratteristici di un flip-flop D hanno gli stessi nomi e significati simili a quelli del latch CD, con la differenza che fanno riferimento al fronte di salita del clock.

\begin{description}
    \item[Tempo di set-up $t_{su}$] tempo minimo in cui $D$ deve essere costante prima del fronte
    \item[Tempo di hold $t_h$] tempo minimo in cui $D$ deve essere costante dopo il fronte
    \item[Tempo di risposta $t_r$] tempo massimo di durata del transitorio sulle uscite $Q$ e $Q'$ dopo il fronte
\end{description}

\noindent
Come nel latch CD, $D$ deve rimanere costante sia durante $t_{su}$, sia durante $t_h$.
Transitorio più lungo di un latch CD, perciò viene usato solo quando serve la sua robustezza, per esempio nelle retroazioni di reti sequenziali sincrone.

\subsubsection{FF-D come elemento di ritardo}

Se all'ingresso $CLK$ viene inviato un segnale periodico, il FF-D ritarda il segnale di uscita $Q$, rispetto al segnale $D$, di un tempo pari al periodo di clock $T$.
Il clock divide il tempo in intervalli discreti tra cui vale la relazione:

$$
Q^{n+1} = D^{n}
$$

\noindent
Questa è una caratteristica che lo rende fondamentale nella realizzazione di reti sequenziali sincrone.

\end{document}
