\documentclass{subfiles}

\usepackage{amsfonts}
\usepackage{amssymb}
\usepackage{amsmath}

\begin{document}

\section{Alee}

\subsection{Comportamento a regime e in transitorio}

Per capire i vincoli di corretto impiego di una RSA, dobbiamo approfondire come risponde ai cambiamenti degli ingressi una rete combinatoria.
Esistono due tipi di comportamenti:

\begin{description}
    \item[In transitorio] che descrive la fase successiva al cambiamento dei segnali di ingresso di una rete combinatoria, caratterizzato dal fatto che l'uscita di essa non ha ancora presentato il valore previsto per la nuova configurazione
    \item[A regime] ovvero il comportamento al termine della fase transitoria
\end{description}

\subsection{Ritardo puro}

Il caso più semplice è di comportamento transitorio, la dove al variare di un ingresso, che provoca la variazione di un'uscita di una rete combinatoria, la rete mantiene il precedente valore dell'uscita per il tempo di propagazione $T_p$ della rete combinatoria, prima di sostituirla col nuovo valore.
Questo transitorio è ineliminabile, ma non dannoso.

\subsection{Alea dinamica}

L'uscita, in regime di transitorio, varia più volte prima di assestarsi sul nuovo valore.
Questo malfunzionamento è causato dai diversi ritardi di propagazione dei percorsi che agiscono sull'uscita $Z$.
Una rete combinatoria descritta da espressioni SP o PS non presenta mai alee dinamiche.

\subsection{Alea statica}

L'uscita dovrebbe rimanere costante, ma subisce, in regime di transitorio, una temporanea variazione.\\

\noindent
esempio:\\

\noindent
I due ingressi di un gate OR cambiano valore contemporaneamente.
Nel caso ideale, l'uscita rimane costante, ma per motivazioni legate al contesto (la disposizione del circuito, le caratteristiche fisiche dei gate, $\dots$), si verifica una minuscola variazione indesiderata dell'uscita.
Per evitare alee statiche in una rete combinatoria, è necessario, ma non sufficiente, variare un solo ingresso alla volta.
Un esempio di non sufficienza nell'evitare le alee statiche è il caso del multiplexer.
Usando l'espressione minima SP per un multiplexer, il cambio di $A$, per colpa del disallineamento dei componenti, può provocare alea statica (nello specifico, a causa del ritardo dei NOT).
Per evitare a priori l'insorgere di alee statiche, quando si effettua la sintesi con le mappe di Karnaugh, si deve scegliere un'espressione ridondante (quindi non minima) che racchiuda in uno stesso raggruppamento rettangolare ogni coppia di $1$ o $0$ adiacenti.\\

\begin{center}
\begin{tabular}{ |c|c|c|c|c| }
\hline
$A \backslash I_1I_0$ & 00 & 01 & 11 & 10 \\
\hline
\hline
0 & 0 & 1 & 1 & 0 \\
1 & 0 & 0 & 1 & 1 \\
\hline
\end{tabular}
\end{center}

\begin{itemize}
    \item espressione SP minima: $A'I_0 + AI_1$
    \item espressione SP priva di alee: $A'I_0 + AI_1 + I_1I_0$
\end{itemize}

\begin{center}
\begin{tabular}{ |c|c|c|c|c| }
\hline
$c \backslash ab$ & 00 & 01 & 11 & 10 \\
\hline
\hline
0 & 0 & 1 & 0 & 0 \\
1 & 0 & 1 & 1 & 1 \\
\hline
\end{tabular}
\end{center}

\begin{itemize}
    \item espressione SP minima: $(a+b)(a'+c)$
    \item espressione SP priva di alee: $(a+b)(a'+c)(b+c)$
\end{itemize}

\begin{center}
\begin{tabular}{ |c|c|c|c|c| }
\hline
$cd \backslash ab$ & 00 & 01 & 11 & 10 \\
\hline
\hline
00 & 0 & 1 & 0 & 0 \\
01 & 0 & 1 & 1 & 1 \\
11 & 0 & 1 & 0 & 0 \\
10 & 0 & 1 & 1 & 1 \\
\hline
\end{tabular}
\end{center}

\begin{itemize}
    \item espressione SP minima: $a'd + ad' + ab + c'd$
    \item espressione SP priva di alee: $a'd + ad' + ab + c'd + bd + ac'$
\end{itemize}

\end{document}
