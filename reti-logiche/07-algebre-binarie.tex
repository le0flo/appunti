\documentclass{subfiles}

\usepackage{amsfonts}
\usepackage{amssymb}
\usepackage{amsmath}

\begin{document}

\section{Algebre binarie}

Si definisce algebra binaria, un sistema matematico formato da un insieme di operatori definiti assiomaticamente ed atti a descrivere con una espressione ogni possibile funzione di variabili binarie.
Oggi è possibile rappresentare ogni funzione binaria con soltanto operatori NAND ($\uparrow$) o NOR ($\downarrow$).

\subsection{NAND}

\begin{description}
    \item[NOT] $x' = x \uparrow x$
    \item[AND] $xy = (x \uparrow y)'$
    \item[OR] $x+y = x' \uparrow y'$
\end{description}

\subsection{NOR}

\begin{description}
    \item[NOT] $x' = x \downarrow x$
    \item[AND] $x + y = (x \downarrow y)'$
    \item[OR] $xy = x' \downarrow y'$
\end{description}

\subsection{Sintesi con NAND}

La sintesi "a NAND" può essere effettuata trasformando un'espressione SP (SP, SPS, SPSP, $\dots$) che descrive la funzione assegnata, in una nuova espressione contenente esclusivamente operatori "$\uparrow$".

\begin{enumerate}
    \item Si parte dall'espressione SP, SPS. SPSP, $\dots$
    \item Si sostituisce il simbolo "$\uparrow$" ad ogni simbolo "$*$"
    \item Si sostituisce il simbolo "$\uparrow$" ad ogni simbolo "$+$" e si completano le variabili singole affiancate a tale simbolo senza aggiungere o togliere parentesi
    \item (Opzionale) Se compaiono segnali di ingresso in forma negata e non sono disponibili, si sostituiscono con il NAND del segnale in forma vera
\end{enumerate}

\subsection{Sintesi con NOR}

\begin{enumerate}
    \item Si parte da un’espressione PS, PSP, PSPS, $\dots$
    \item Si sostituisce il simbolo "$\downarrow$" ad ogni simbolo "$+$"
    \item Si sostituisce il simbolo "$\downarrow$" ad ogni simbolo "$*$" e si completano le variabili singole affiancate a tale simbolo senza aggiungere o togliere parentesi
    \item (Opzionale) Se compaiono segnali di ingresso in forma negata e non sono disponibili, li sostituisco con il NOR del segnale in forma vera
\end{enumerate}

\end{document}
