\documentclass{subfiles}

\usepackage{amsfonts}
\usepackage{amssymb}
\usepackage{amsmath}

\begin{document}

\section{Reti combinatorie}

\subsection{Definizione}

Una rete logica la dove l'uscita dipende unicamente dagli ingressi correnti.

\subsection{Comportamento}

La tabella della verità, con tutte le combinazioni di ingressi possibili e i rispettivi segnali d'uscita.

\subsection{Struttura}

I componenti che realizzano la rete logica indicata. Può essere dichiarata sotto forma di:

\begin{description}
	\item[Espressione] esempio: $z = (x \equiv y)(x \oplus w)$
	\item[Schema logico]
\end{description}

\subsection{Dalle tabelle di verità a espressioni}

Uno dei metodi per passare da una tabella di comportamento ad una espressione, è mediante l'uso delle così dette espressioni canoniche.

\subsection{Espressioni canoniche}

\begin{description}
	\item[Espressione canonica SP] Somma di prodotti, prima forma canonica
	\item[Espressione canonica PS] Prodotto di somme, seconda forma canonica
\end{description}

\subsubsection{Full adder}

Una rete logica con 3 ingressi ($a,b,r$) e due uscite ($S,R$). Questo rappresenta:

\begin{description}
	\item[$S = 1$] quando il numero di uno dei suoi ingressi è dispari
	\item[$R = 1$] quando in ingresso ci sono due o più 1
\end{description}

\noindent
Questa rete è combinatoria perchè l'uscita dipende solo dagli ingressi attuali.
Sarà una rete fondamentale per realizzare operazioni aritmetiche tra numeri binari.

\begin{center}
\begin{tabular}{ |c|c|c|c|c| }
\hline
$a$ & $b$ & $r$ & $S$ & $R$ \\
\hline
\hline
0 & 0 & 0 & 0 & 0 \\
0 & 0 & 1 & 1 & 0 \\
0 & 1 & 0 & 1 & 0 \\
0 & 1 & 1 & 0 & 1 \\
1 & 0 & 0 & 1 & 0 \\
1 & 0 & 1 & 0 & 1 \\
1 & 1 & 0 & 0 & 1 \\
1 & 1 & 1 & 1 & 1 \\
\hline
\end{tabular}
\end{center}

\noindent
Per trovare un'espressione, prendiamo come esempio l'output $S$:

$$
S = 1 \implies Riga_1, Riga_2, Riga_4, Riga_7
$$

\noindent
Perciò $S = 1$ quando:

\begin{itemize}
    \item $C_1 = 1 \implies a = 0$, $b = 0$, $r = 0$
    \item $C_2 = 1 \implies a = 0$, $b = 1$, $r = 0$
    \item $C_4 = 1 \implies a = 1$, $b = 0$, $r = 0$
    \item $C_7 = 1 \implies a = 1$, $b = 1$, $r = 1$
\end{itemize}

\noindent
Che possiamo trasformare in:

\begin{itemize}
\item $C_1 = 1 \implies a' = 1$, $b' = 1$, $r' = 1$
\item $C_2 = 1 \implies a' = 1$, $b = 1$, $r' = 1$
\item $C_4 = 1 \implies a = 1$, $b' = 1$, $r' = 1$
\item $C_7 = 1 \implies a = 1$, $b = 1$, $r = 1$
\end{itemize}

\noindent
Che ci permette di ottenere:

$$
S = C_1 + C_2 + C_4 + C_7 = a'b'r' + a'br' + ab'r' + abr
$$

\noindent
Usando lo stesso ragionamento otteniamo che:

$$
R = a'br + ab'r + abr' + abr
$$

\subsection{Notazione simbolica}

\begin{description}
	\item[Mintermine] il vettore dei bit, rappresentati mediante il loro indice, che assumono il valore 1
	\item[Maxtermine] il vettore dei bit, rappresentati mediante il loro indice, che assumono il valore 0
\end{description}

\noindent
Mintermine e maxtermine sono complementari tra loro.
Una configurazione è o un mintermine o un maxtermine.
Il pedice degli operatori $\Sigma$ e $\Pi$ corrisponde al numero di ingressi che danno origine al mintermine o maxtermine.\\

\noindent
esempio, Full adder:

\begin{center}
\begin{tabular}{ |c|c|c|c|c|c| }
\hline
$i$ & $a$ & $b$ & $r$ & $S$ & $R$ \\
\hline
\hline
0 & 0 & 0 & 0 & 0 & 0 \\
1 & 0 & 0 & 1 & 1 & 0 \\
2 & 0 & 1 & 0 & 1 & 0 \\
3 & 0 & 1 & 1 & 0 & 1 \\
4 & 1 & 0 & 0 & 1 & 0 \\
5 & 1 & 0 & 1 & 0 & 1 \\
6 & 1 & 1 & 0 & 0 & 1 \\
7 & 1 & 1 & 1 & 1 & 1 \\
\hline
\end{tabular}
\end{center}

\begin{itemize}
    \item $S(a,b,r) = \Sigma_3 m(1,2,4,7)$
    \item $S(a,b,r) = \Pi_3 M(0,3,5,6)$
    \item $R(a,b,r) = \Sigma_3 m(3,5,6,7)$
    \item $R(a,b,r) = \Pi_3 M(0,1,2,4)$
\end{itemize}

\subsection{Equivalenza tra espressioni}

L'equivalenza tra due espressioni si indica con:

$$
E_1 = E_2
$$

\noindent
Si può definire tale se e solo se le espressioni esprimono la stessa funzione (o TdV).

\subsection{Funzioni incomplete}

Anche delle espressioni che forniscono una valutazione uguale, limitate al dominio di una funzione incompleta, sono dette equivalenti.
Un esempio di funzione incompleta è un encoder da $1$ a $N$ bit.
Prendendo come esempio più specifico, un encoder con 3 ingressi, a seconda del valore che assumono i bit di uscita delle configurazioni non rilevanti (fuori dal dominio della funzione), l'espressione risultante cambia.

\subsection{Equivalenze notevoli}

Proprietà della somma e del prodotto logico:

\begin{description}
    \item[Commutativa] $x+y = y+x$ e $x*y = y*x$
    \item[Associativa] $(x+y)+z = x+y+z$ e $(x*y)*z = x*y*z$ (utile per ridurre il fan-in)
    \item[Distributiva] $(x*y) + (x*z) = x*(y+z)$ e $(x+y) * (x+z) = x+(y*z)$ (valida soltanto in algebra binaria)
    \item[Idempotenza] $x+x = x$ e $x*x = x$
    \item[Identità] $x+0 = x$ e $x*1 = x$
    \item[Limite] $x+1 = 1$ e $x*0 = 0$
    \item[Involuzione] $(x')' = x$ (viene usata per amplificare il segnale)
    \item[Limitazione] $x+x' = 1$ e $x * x' = 0$
    \item[Combinazione] $xy + xy' = x$ e $(x+y) * (x+y') = x$
    \item[Prima legge di De Morgan] $(x+y)' = x'*y'$
    \item[Seconda legge di De Morgan] $(x*y)' = x'+y'$
    \item[Consenso] $xy+x'z+yz = xy+x'z$ e $(x+y)*(x'+z)*(y+z) = (x+y)*(x'+z)$
\end{description}

\subsection{Manipolazione algebrica di espressioni}

Data l'espressione per il ritorno di un full adder:

$$
R = a'br + ab'r + abr' + abr
$$

\noindent
Essa può essere semplificata usando i seguenti passaggi:

\begin{description}
    \item[Distribuzione] $a'br + ab'r + ab*(r'+r)$
    \item[Limitazione] $a'br + ab'r + ab1$
    \item[Identità] $a'br + ab'r + ab$
\end{description}

\noindent
Oppure una versione ancora più semplificata è:

\begin{description}
    \item[Idempotenza] $a'br + ab'r + abr' + abr + abr + abr$
    \item[Distribuzione] $br*(a'+a) + ar*(b'+b) + ab*(r'+r)$
    \item[Limitazione] $br1 + ar1 + ab1$
    \item[Identità] $br + ar + ab$
\end{description}

\noindent
Rispetto all'espressione originale $R = a'br + ab'r + abr' + abr$, da una rete formata da:

\begin{itemize}
    \item 1 OR da 4 ingressi
    \item 4 AND da 3 ingressi
    \item 3 NOT
\end{itemize}

\noindent
Siamo passati, usando l'espressione $R = br + ar + ab$, ad una rete composta da:

\begin{itemize}
    \item 1 OR a 3 ingressi
    \item 3 AND da 2 ingressi
\end{itemize}

\subsection{Il problema della sintesi}

La sintesi è il processo per individuare l'espressione "migliore" per la realizzazione della funzione assegnata.
"Migliore" può essere definito con criteri anche opposti tra loro, come:

\begin{itemize}
    \item Rapidità di progetto
    \item Massima flessibilità
    \item Massima velocità
    \item Minima complessità
\end{itemize}

\noindent
Abbiamo due possibili obbiettivi:

\begin{description}
	\item[Reti di costo minimo] Massima velocità e Minima complessità
	\item[Reti programmabili] Rapidità di progetto e Massima flessibilità
\end{description}

\end{document}
