\documentclass{article}

\usepackage{amsfonts}
\usepackage{amssymb}
\usepackage{amsmath}

\begin{document}

\section{Registro}

Un registro a $k$ bit è una rete logica sincrona in grado di memorizzare un dato formato da $k$ bit.
La rete ad ogni fronte di salita del clock, memorizza e rende disponibile sulle uscite $OUT[k-1 \dots 0]$ il dato $IN[k-1 \dots 0]$ in ingresso se l'ingresso $WE = 1$ (write enable) mentre mantiene il valore precedentemente memorizzato se $WE = 0$.
Inoltre la rete è dotata di un ingresso asincrono $A\_RESET$ che, se $1$, pone a livello logico $0$ tutti i bit del registro, indipendentemente dal valore dei segnali $WE$, $IN$ e del clock.
$WE$ è un comando sincrono, agisce al fronte del clock, quindi al termine degli intervalli di tempo in cui vale $1$.

\subsection{Varianti}

Molto spesso esistono più varianti delle RSS notevoli, che si differenziano per:

\begin{itemize}
    \item comandi disponibili (registri con o senza $A\_RESET$)
    \item se i comandi disponibili sono sincroni o asincroni
    \item per l'ordine di priorità dei comandi (i comandi asincroni sono sempre prioritari rispetto a quelli sincroni)
\end{itemize}

\noindent
Bisogna sapere quale ingresso sincrono è prioritario se vi sono più ingressi sincroni.

\subsection{Flip-flop T}

Usando un FF-D, il flip-flop T (toggle) è realizzato prendendo l'uscita ($Q$) e il suo complemento ($Q'$) dal FF-D e collegarle ad un selettore che a sua volta è collegato in retroazione diretta all'ingresso $D$ del FF-D.

\subsection{Shift register}

Uno shift register (o registro a scorrimento) a $k$ bit è una rete in grado di memorizzare gli ultimi $k$ bit ricevuti su un segnale di ingresso seriale $IN$, rendendoli disponibili attraverso le uscite $OUT[k-1 \dots 0]$.
Tutti i bit memorizzato possono essere portati a $0$ asserendo l'ingresso asincrono $A\_RESET$.
Questo componente è utile per:

\begin{itemize}
    \item Ritardare da $1$ a $k$ intervalli di tempo il segnale $IN$.
    \item Riconoscere il verificarsi di stringe d'ingresso
    \item Convertitore Seriale/Parallelo e Parallelo/Seriale
    \item Conteggio
    \item Rotazione verso destra o sinistra
    \item Moltiplicazione o divisione per una potenza di 2
\end{itemize}

\subsubsection{Comandi}

\begin{description}
    \item[$EN$ o enable] l'ingresso $IN$ viene memorizzato solo se $EN=1$ e le uscite non cambiano quando $EN=0$
    \item[$LD$ o load] se $LD=1$ il valore memorizzato diventa $IN$
    \item[$R/L'$ o destra/sinistra'] consente variare la direzione dello shift
\end{description}

\subsubsection{Universal shift register}

Quando tutti i comandi elencati precedentemente sono disponibili, si parla di universal shift register.
4 comandi possibili:

\begin{description}
    \item[Hold] corrispondente a $LD=0$, $EN=0$, $R/L'=-$
    \item[Shift right] corrispondente a $LD=0$, $EN=1$, $R/L'=1$
    \item[Shift left] corrispondente a $LD=0$, $EN=1$, $R/L'=0$
    \item[Load] corrispondente a $LD=1$, $EN=-$, $R/L'=-$
\end{description}

\noindent
Sono codificabili da 2 bit di comando $S_1$ e $S_0$.
Posso avere un solo ingresso $IN$ che entra a sinistra o a destra a seconda del valore dei bit di comando, o due ingressi diversi $IN\_R$ e $IN\_L$.

\end{document}
