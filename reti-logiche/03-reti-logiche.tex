\documentclass{subfiles}

\usepackage{amsfonts}
\usepackage{amssymb}
\usepackage{amsmath}

\begin{document}

\section{Reti logiche}

Una rete logica è un'astrazione che rappresenta una combinazione di "interruttori" che elaborano segnali binari.

\subsection{Gate}

Definiamo "gate" tutti i componenti elementari di cui non conosciamo il come sono fatti, ma il loro comportamento.
Il numero di funzioni diverse di $n$ ingressi binari con un'uscita binaria è:

$$
2^{2^n}
$$

\noindent
I componenti elementari, o funzioni possibili, limitandosi ai componenti con un unico segnale binario di ingresso $x$ sono $4, (n = 1)$.\\

\noindent
esempio:

\begin{center}
\begin{tabular}{ |c|c|c|c|c| }
\hline
$x$ & $f_1$ & $f_2$ & $f_3$ & $f_4$ \\
\hline
\hline
0 & 0 & 0 & 1 & 1 \\
1 & 0 & 1 & 0 & 1 \\
\hline
\end{tabular}
\end{center}

\noindent
Ogni gate è descritto da:

\begin{description}
	\item[Tabella della verità] con ogni riga che riporta un possibile ingresso e la corrispondente uscita
	\item[Simbolo circuitale] per rappresentarlo graficamente e distinguerlo
	\item[Espressione] un modo di rappresentare la relazione tra ingressi ed uscite
\end{description}

\subsubsection{Gate NOT}

Tabella della verità:

\begin{center}
\begin{tabular}{ |c|c| }
\hline
$x$ & $y$ \\
\hline
\hline
0 & 1 \\
1 & 0 \\
\hline
\end{tabular}
\end{center}

\noindent
Espressione: $y = \overline{x}$ oppure $y = x'$

\subsubsection{Gate AND}

Tabella della verità:

\begin{center}
\begin{tabular}{ |c|c|c| }
\hline
$x$ & $y$ & $z$ \\
\hline
\hline
0 & 0 & 0 \\
0 & 1 & 0 \\
1 & 0 & 0 \\
1 & 1 & 1 \\
\hline
\end{tabular}
\end{center}

\noindent
Espressione: $z = x*y$ oppure $z = xy$

\subsubsection{Gate OR}

Tabella della verità:

\begin{center}
\begin{tabular}{ |c|c|c| }
\hline
$x$ & $y$ & $z$ \\
\hline
\hline
0 & 0 & 0 \\
0 & 1 & 1 \\
1 & 0 & 1 \\
1 & 1 & 1 \\
\hline
\end{tabular}
\end{center}

\noindent
Espressione: $z = x + y$

\subsubsection{Gate EXOR/XOR}

Tabella della verità:

\begin{center}
\begin{tabular}{ |c|c|c| }
\hline
$x$ & $y$ & $z$ \\
\hline
\hline
0 & 0 & 0 \\
0 & 1 & 1 \\
1 & 0 & 1 \\
1 & 1 & 0 \\
\hline
\end{tabular}
\end{center}

\noindent
Espressione: $z = x \oplus y$ \\

\noindent
lo XOR viene anche detto somma modulo 2, in quando il suo output può essere interpretato come il risultato della somma di due bit, escludendo il riporto.

\subsubsection{Gate NAND}

Tabella della verità:

\begin{center}
\begin{tabular}{ |c|c|c| }
\hline
$x$ & $y$ & $z$ \\
\hline
\hline
0 & 0 & 1 \\
0 & 1 & 1 \\
1 & 0 & 1 \\
1 & 1 & 0 \\
\hline
\end{tabular}
\end{center}

\noindent
Espressione: $z = x \uparrow y$ oppure $z = \overline{xy}$

\subsubsection{Gate NOR}

Tabella della verità:

\begin{center}
\begin{tabular}{ |c|c|c| }
\hline
$x$ & $y$ & $z$ \\
\hline
\hline
0 & 0 & 1 \\
0 & 1 & 0 \\
1 & 0 & 0 \\
1 & 1 & 0 \\
\hline
\end{tabular}
\end{center}

\noindent
Espressione: $z = x \downarrow y$ oppure $z = \overline{x+y}$

\subsubsection{Gate EXNOR}

Tabella della verità:

\begin{center}
\begin{tabular}{ |c|c|c| }
\hline
$x$ & $y$ & $z$ \\
\hline
\hline
0 & 0 & 1 \\
0 & 1 & 0 \\
1 & 0 & 0 \\
1 & 1 & 1 \\
\hline
\end{tabular}
\end{center}

\noindent
Espressione: $z = x \equiv y$ oppure $z = \overline{x \oplus y}$

\subsubsection{AND e OR con $n$ ingressi}

esempio, AND con $n=3$:

\begin{center}
\begin{tabular}{ |c|c|c|c| }
\hline
$x$ & $y$ & $w$ & $z$ \\
\hline
\hline
0 & 0 & 0 & 0 \\
0 & 0 & 1 & 0 \\
0 & 1 & 0 & 0 \\
0 & 1 & 1 & 0 \\
1 & 0 & 0 & 0 \\
1 & 0 & 1 & 0 \\
1 & 1 & 0 & 0 \\
1 & 1 & 1 & 1 \\
\hline
\end{tabular}
\end{center}

\subsubsection{Tabelle per gate con 2 ingressi}

\begin{center}
\begin{tabular}{ |c|c|c|c|c|c|c|c|c|c|c|c|c|c|c|c|c|c| }
\hline
$x$ & $y$ & $f_0$ & $f_1$ & $f_2$ & $f_3$ & $f_4$ & $f_5$ & $f_6$ & $f_7$ & $f_8$ & $f_9$ & $f_{10}$ & $f_{11}$ & $f_{12}$ & $f_{13}$ & $f_{14}$ & $f_{15}$ \\
\hline
\hline
0 & 0 & 0 & 0 & 0 & 0 & 0 & 0 & 0 & 0 & 1 & 1 & 1 & 1 & 1 & 1 & 1 & 1 \\
0 & 1 & 0 & 0 & 0 & 0 & 1 & 1 & 1 & 1 & 0 & 0 & 0 & 0 & 1 & 1 & 1 & 1 \\
1 & 0 & 0 & 0 & 1 & 1 & 0 & 0 & 1 & 1 & 0 & 0 & 1 & 1 & 0 & 0 & 1 & 1 \\
1 & 1 & 0 & 1 & 0 & 1 & 0 & 1 & 0 & 1 & 0 & 1 & 0 & 1 & 0 & 1 & 0 & 1 \\
\hline
\end{tabular}
\end{center}

\subsubsection{Bus di segnali}

Un gruppo di segnali viene detto bus.
Per indicare un bus di $n$ segnali che codificano un'informazione, si usa la notazione con parentesi quadre:

$$
\text{TEST}[n-1 \dots 0]
$$

\noindent
esempio:

\begin{itemize}
	\item bus a $3$ bit per il colore: $\text{COLORE}[2 \dots 0]$
\end{itemize}

\noindent
Per riferirci ad uno dei segnali del bus, si usa la notazione:

$$
\text{TEST}0, \dots, \text{TEST}n-1
$$

\noindent
esempio:

\begin{itemize}
	\item secondo segnale del bus dei colori: $\text{COLORE}1$
\end{itemize}

\subsubsection{Ritardo di propagazione}

Pur lavorando con componenti astratti, bisogna tenere in considerazione il fatto che possono essere componenti reali.
La differenza principale tra i due, da prendere in considerazione, è il "ritardo di propagazione", indicato con $\tau_p$, ed esprime il tempo che un segnale impiega per completare la transizione tra stati.
Un impulso di durata inferiore a $\tau_p$ su uno degli ingressi non appare in uscita.

\end{document}
