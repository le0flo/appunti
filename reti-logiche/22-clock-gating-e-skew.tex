\documentclass{subfiles}

\usepackage{amsfonts}
\usepackage{amssymb}
\usepackage{amsmath}

\begin{document}

\section{Clock gating e clock skew}

\subsection{Clock gating}

Quando un segnale viene usato per "fermare" il clock, questo viene definito come clock gating.
Un rischio del clock gating è che, se in presenza di alee, introdotte dalla rete responsabile per il segnale di attivazione del clock, possono verificarsi fronti di salita spuri del segnale uscente dal clock gating ($CK\_G$).
Per evitare questo, bisogna assicurarsi che sull'ingresso clock di un FF-D vengano mandati soltanto segnali sincroni (ovvero che commutano una sola volta all'inizio del periodo di clock).

\subsection{Clock skew}

Il clock gating, oltre a generare potenziali glitch, può portare anche al fenomeno del clock skew.
Il clock skew è un ritardo che si forma a causa del ritardo introdotto dalla rete combinatoria che viene utilizzata per il clock gating.
Questo è potenzialmente dannoso poiché la RSS non sincronizzata con il clock potrebbe aggiornare il suo stato con nuovi valori prodotti dalla RSS sincronizzata (che potrebbe essere anche in regime di transitorio).

\end{document}
