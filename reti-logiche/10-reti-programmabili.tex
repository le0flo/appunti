\documentclass{subfiles}

\usepackage{amsfonts}
\usepackage{amssymb}
\usepackage{amsmath}

\begin{document}

\section{Reti programmabili}

\subsection{LSI e VLSI}

Rispettivamente Large Scale Integration e Very Large Scale Integration.
Appartengono a queste categorie i chip composti da milioni di gate.
Sono molto costosi da produrre e per ammortizzare le spese necessitano di essere prodotti e, soprattutto, venduti in massa.

\subsection{Reti combinatorie programmabili}

Una rete combinatoria in grado di presentare diverse relazioni ingresso/uscita selezionabili mediante l'attribuzione di una determinata configurazione di valori ad un gruppo di segnali interni, detti bit di programmazione.

\subsection{Tabella della verità come memoria non modificabile}

Le reti programmabili possono essere usate per realizzare reti combinatorie a partire da memorie immutabili.
Una memoria è un circuito che realizza fisicamente un array, ovvero una lista di dati indicizzabile con un intero (indirizzo).
Una tabella della verità può essere interpretata come una memoria non modificabile, ovvero una lista di $2^n$ dati immutabili, tra cui gli $n$ ingressi mi permettono di scegliere.

\subsubsection{Memoria di sola lettura, ROM}

Una Read Only Memory è un circuito che realizza una memoria a sola lettura, ovvero che contiene ad ogni indirizzo un dato fissato.
Esse sono dunque delle reti programmabili in grado di realizzare qualsiasi funzione di $n$ variabili.
Le ROM sono programmabili una sola volta (OTP, one time programmable) e sono quindi memorie di sola lettura non volatili.

\subsubsection{Memorie non volatili, a sola lettura e programmabili}

\begin{description}
    \item[ROM] programmabile una volta, soltanto dal costruttore, fuori dal circuito
    \item[PROM] programmabile una volta, dall'utente, fuori dal circuito
    \item[EPROM] programmabile più volte, dall'utente, fuori dal circuito
    \item[EEPROM / FLASH] programmabile più volte, dall'utente, dentro il circuito
\end{description}

\subsubsection{CPLD e FPGA}

Le EEPROM sono un esempio di CLB, Configurable Logic Block.
Cambiando i bit di programmazione possiamo sintetizzare funzioni diverse.
La loro evoluzione ha portato a rendere disponibili delle versioni più avanzate, come i Complex PLD e gli FPGA, Field Programmable Gate Array.
In un FPGA, anche le connessioni tra CLB sono programmabili.

\subsection{Hardware description language, HDL}

I circuiti programmabili e le connessioni in CPLD e FPGA sono configurati a partire da file scritti in linguaggi di descrizione hardware.
Sono dei linguaggi di programmazione a tutti gli effetti, specializzati per gestire la concorrenza intrinseca a livello hardware e la nozione di tempo.
I progetti sono poi passati al sintetizzatore, che realizza una sintesi mappabile sui gate, le altre risorse e le connessioni disponibili in uno specifico FPGA.
I due principali HDL sono:

\begin{itemize}
    \item VHDL
    \item Verilog
\end{itemize}

\end{document}
