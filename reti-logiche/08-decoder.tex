\documentclass{subfiles}

\usepackage{amsfonts}
\usepackage{amssymb}
\usepackage{amsmath}

\begin{document}

\section{Decoder}

esempio:\\

\noindent
Una rete combinatoria che converte un numero binario a 3 bit a codice 1 su 8 ($2^3$).
La convenzione per indicare gli ingressi è usare le lettere maiuscole a partire dalla A, la quale rappresenta l'ingresso con la cifra meno significativa.
Ogni uscita ha solo una configurazione per cui essa vale "$1$", ovvero quella che codifica il numero dell'uscita stessa.
Data la proposizione precedente, è chiaro che la sintesi canonica minima SP prevede quell'unico mintermine nell'espressione.

\subsection{Decoder (DEC) generico $n:2^n$}

Una rete che transcodifica un numero binario a $n$ bit a codice $1$ su $2^n$.
Gli $n$ ingressi vengono anche indicati come indirizzi ($A$ addresses), con $A_0$ indirizzo di minor peso.
L'indice $i$ dell'uscita $U_i$ attivata è pari al numero rappresentato dalla configurazione binaria degli stessi ingressi.

$$
i = A_{n-1} * 2^{n-1} + \dots + A_1 * 2^1 + A_0 * 2^0
$$

\begin{center}
\begin{tabular}{ |c|c|c|c|c| }
\hline
$A_{n-1}$ & $\dots$ & $A_2$ & $A_1$ & $A_0$ \\
\hline
\hline
0 & $\dots$ & 0 & 0 & 0 \\
0 & $\dots$ & 0 & 0 & 1 \\
0 & $\dots$ & 0 & 1 & 0 \\
0 & $\dots$ & 0 & 1 & 1 \\
$\dots$ & $\dots$ & $\dots$ & $\dots$ & $\dots$ \\
1 & $\dots$ & 1 & 1 & 1 \\
\hline
\end{tabular}
\end{center}

\begin{center}
\begin{tabular}{ |c|c|c|c|c| }
\hline
$U_{2^n-1}$ & $\dots$ & $U_2$ & $U_1$ & $U_0$ \\
\hline
\hline
0 & $\dots$ & 0 & 0 & 1 \\
0 & $\dots$ & 0 & 1 & 0 \\
0 & $\dots$ & 1 & 0 & 0 \\
0 & $\dots$ & 0 & 0 & 0 \\
$\dots$ & $\dots$ & $\dots$ & $\dots$ & $\dots$ \\
1 & $\dots$ & 0 & 0 & 0 \\
\hline
\end{tabular}
\end{center}

\subsection{Effetto di carico: fan-out}

Il fan-out è il numero massimo di gate che possono collegarsi all'ingresso di un singolo gate.
La tecnologia di oggi ci permette di avere numero di fan-out $\geq 10$, ma rimane comunque un fattore importante da considerare.

\subsection{MSI e LSI}

\begin{description}
    \item[MSI] Medium Scale Integration
    \item[LSI] Large Scale Integration
\end{description}

\noindent
Esistono DEC in forma integrata con 2, 3, o 4 bit di indirizzo.
Questi ricadono tutti sotto l'ala della MSI.

\end{document}
