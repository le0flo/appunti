\documentclass{subfiles}

\usepackage{amsfonts}
\usepackage{amssymb}
\usepackage{amsmath}

\begin{document}

\section{Sistemi}

Segue un esempio di sistema lineare generico a coefficienti in $\mathbb{K}$:\\

\noindent
Dati $a_{ij} \in \mathbb{K}, b_i \in \mathbb{K}$:

$$
S: \begin{cases}
a_{11} x_1 + \dots + a_{1n} x_n = b_1\\
\vdots\\
a_{m1} x_1 + \dots + a_{mn} x_n = b_m
\end {cases}
$$

\noindent
Un sistema lineare può essere:

\begin{description}
    \item[Omogeneo] $b_1, \dots, b_m = 0$
    \item[Non omogeneo] $\exists \ b \neq 0$
\end{description}

\noindent
L'insieme delle soluzioni viene indicato come:

$$
W_S = \{ \vec{x} \in \mathbb{K}^n : S(\vec{x}) = 0 \}
$$

\noindent
Semplificare un sistema significa ottenere un altro insieme di soluzioni $W_{S'}$ tale che:

$$
W_{S'} = W_S
$$

\subsection{Mosse di Gauss}

\begin{enumerate}
    \item Scambiare due righe: $R_i \leftrightarrow R_j$
    \item Moltiplicare una riga per uno scalare: $R_i \to \lambda R_i : \lambda \in \mathbb{K}$
    \item Sostituire una riga con se stessa sommata ad un multiplo di un'altra riga: $R_i \to R_i + \lambda R_j : \lambda \in \mathbb{K}$
\end{enumerate}

Applicando le 3 mosse di Gauss ad un sistema lineare $S$, deriviamo il sistema $S'$ con il medesimo numero di soluzioni rispetto al sistema originale $S$.

\subsection{Matrice dei coefficienti}

$$
A_S = A =
\begin{pmatrix}
a_{11} & \dots & a_{1n}\\
\vdots & \ddots & \vdots\\
a_{m1} & \dots & a_{mn}
\end{pmatrix}
\in M(m,n,\mathbb{K})
$$

\subsection{Vettore dei termini noti}

$$
\vec{b} = (b_1, \dots, b_m) \in \mathbb{K}^m
$$

\subsection{Matrice del sistema}

$$
C_S = C = ( A | \vec{b} ) =
\begin{pmatrix}
a_{11} & \dots & a_{1n} & b_1\\
\vdots & \ddots & \vdots & \vdots\\
a_{m1} & \dots & a_{mn} & b_m
\end{pmatrix}
\in M(m,n,\mathbb{K})
$$

\subsection{Sistema lineare omogeneo associato}

$$
S_0: \begin{cases}
a_{11} x_1 + \dots + a_{1n} x_n = 0\\
\vdots\\
a_{m1} x_1 + \dots + a_{mn} x_n = 0
\end {cases}
$$

$$
W_S = \vec{x} + W_{S_0}
$$

\subsection{Rango}

Il rango di un sistema $S$ è:

$$
rg(S)= \dim \space span(C^1, \dots, C^n, \vec{b}) \subseteq \mathbb{K}^m
$$

\noindent
Il rango è sempre minore del numero di righe $m$.
Se la matrice del sistema $C$ è in forma di "Gauss-Jordan":

$$
rg(S) = n\text{-pivot}
$$

\noindent
Possiamo dire che:

$$
rg(S) \leq \min(m, n)
$$

\subsection{Teorema di Rouché-Capelli}

Dato un sistema lineare $S$ e il sistema lineare omogeneo associato $S_0$, $S$ ha la soluzione se e soltanto se:

$$
rg(S) = rg(S_0)
$$

\noindent
Se $W_S \neq \emptyset$, allora la $dim(W_S) = n - rg(S_0)$.

\end{document}
