\documentclass{article}

\usepackage{amsfonts}
\usepackage{amssymb}
\usepackage{amsmath}

\begin{document}

\section{Anelli}

Dato l'insieme $\mathbb{A}$ e due operazioni $P_1,P_2: \mathbb{A} \times \mathbb{A} \to \mathbb{A}$ tali che:

\begin{itemize}
    \item $(\mathbb{A}, P_1)$ sia un gruppo abeliano
    \item $P_2$ è associativa
    \item valgono le leggi distributive tra le due operazioni
\end{itemize}

\noindent
esempi:

\begin{itemize}
    \item $(\mathbb{Z}, +, *)$ è un anello
    \item $(\mathbb{R}[x], +, *)$ è un anello
    \item $(\mathbb{M}(n), +, *)$ è un anello
\end{itemize}

\subsection{Condizioni particolari}

\begin{itemize}
    \item se $P_2$ è commutativa, l'anello si dice commutativo
    \item se $P_2$ ha un elemento neutro, $\mathbb{A}$ si dice: "anello con unità"
    \item se $(\mathbb{A} \backslash \vec{0}, P_2)$ è un gruppo, allora $\mathbb{A}$ si dice anello di divisione o con corpo
    \item quando il corpo è commutativo, $\mathbb{A}$ è un *campo*
\end{itemize}

\end{document}
