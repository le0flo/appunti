\documentclass{article}

\usepackage{amsfonts}
\usepackage{amssymb}
\usepackage{amsmath}

\begin{document}

\section{Gruppi}

Dato l'insieme $\mathbb{G}$ ed un'operazione binaria di tipo $P_1: \mathbb{G} \times \mathbb{G} \to \mathbb{G}$, tale che:

\begin{itemize}
    \item $\exists$  un elemento neutro per l'operazione
    \item l'operazione deve essere associativa
    \item $\forall$ elemento di $\mathbb{G}$ ammette un'inverso
\end{itemize}

\noindent
Possiamo definire il gruppo $(\mathbb{G}, P_1)$.\\

\noindent
esempi:

\begin{itemize}
    \item $(\mathbb{Q}, +)$ è un gruppo
    \item $(\mathbb{Q}, *)$ non è un gruppo, poiché $0$ non ha un elemento inverso
    \item $(\mathbb{R}, +)$ è un gruppo
    \item $(\mathbb{C}, +)$ è un gruppo
\end{itemize}

\subsection{Gruppo commutativo}

Un gruppo si dice commutativo (o abeliano) se l'operazione $*_1$ è anche commutativa.

\subsection{Sottogruppi}

Se un sottoinsieme del gruppo $(\mathbb{G}, P_1)$ rispetta le 3 condizioni di esistenza per i gruppi con l'operazione $P_1$, allora esso può essere definito un sottogruppo.

\end{document}
