\documentclass{subfiles}

\usepackage{amsfonts}
\usepackage{amssymb}
\usepackage{amsmath}

\begin{document}

\section{Funzioni}

Dati due insiemi $\mathbb{A}$ e $\mathbb{B}$:

$$
f: \mathbb{A} \to \mathbb{B}
$$

\begin{itemize}
    \item $\mathbb{A} =$ Dominio della funzione
    \item $\mathbb{B} =$ Codominio della funzione
\end{itemize}

\subsection{Composizione di funzioni}

Date le funzioni $f: \mathbb{A} \to \mathbb{B}$ e $g: \mathbb{B} \to \mathbb{C}$:

$$
\mathbb{A} \to \mathbb{C} \implies a \to g(f(a))
$$

\noindent
La funzione composta si indica con: $g o f \longrightarrow$ "$g$ composto da $f$"

\subsection{Classificazione delle funzioni}

Una funzione $f: \mathbb{A} \to \mathbb{B}$ può essere:

\begin{description}
    \item[Iniettiva] $\forall a \in \mathbb{A} : \exists b \in \mathbb{B}: a \to b$
    \item[Suriettiva] $\mathbb{A} = \mathbb{B}$
    \item[Biettive] entrambe le precedenti
\end{description}

\end{document}
