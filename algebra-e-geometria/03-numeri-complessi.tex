\documentclass{subfiles}

\usepackage{amsfonts}
\usepackage{amssymb}
\usepackage{amsmath}

\begin{document}

\section{Numeri complessi}

Un numero complesso viene rappresentato nel seguente modo:

$$
z = a + bi
$$

\begin{itemize}
    \item $a$ è la parte reale
    \item $b$ è la parte immaginaria
    \item $i$ è l'unità immaginaria
\end{itemize}

\begin{description}
    \item $\mathbb{C} =$ insieme dei numeri complessi
    \item $\mathbb{R} \subseteq \mathbb{C}$
\end{description}

\subsection{Operazioni con i numeri complessi}

\begin{description}
    \item[Somma] $(a+bi) + (c+di) = (a+c) + (b+d)i$
    \item[Moltiplicazione] $(a+bi) * (c+di) = (ac - bd) + (ad + bc)i$
\end{description}

\subsection{Simbologia}

\begin{itemize}
    \item $\bar{z}$ è il coniugato del numero reale $z \longrightarrow z = a+bi \implies \bar{z} = a-bi$
    \item $|z|$ è il modulo del numero reale $z \longrightarrow z = a+bi \implies |z| = \sqrt{a^2 + b^2} \in \mathbb{R}$
    \item $z * \bar{z} = (a+bi)(a-bi) = a^2 + b^2 = |z|^2$
    \item $z^{-1}$ è l'inverso del numero reale $z \longrightarrow z = a+bi \implies z^{-1} = \frac{\bar{z}}{|z|^2}$
\end{itemize}

\subsection{Formula di Eulero}

Dati $r \in \mathbb{R}$ e $M \in (0, 2\pi)$:

$$
z = r(\cos(M)) + i r(\sin(M)) \in \mathbb{C} = r(\cos(M) + i\sin(M)) = re^{iM} \implies |z| = r, \bar{z} = re^{-iM}
$$

\noindent
esempio:

\begin{description}
    \item $e^{i\pi} + 1 = 0$
\end{description}

\subsection{Teorema fondamentale dell'algebra}

Ogni polinomio a coefficienti complessi ammette una radice complessa.

$$
p(x) = a_0 + a_1 x + a_2 x^2 + \dots + a_n x^n
$$

\begin{itemize}
    \item $n$ = grado del polinomio
    \item $a \in \mathbb{C}$ = coefficienti
    \item $\forall z \in \mathbb{C}$ possiamo considerare $p(z) = a_0 + a_1 z + a_2 z^2 + \dots + a_n z^n \in \mathbb{C}$
\end{itemize}

\end{document}
