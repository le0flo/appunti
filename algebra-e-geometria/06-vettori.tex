\documentclass{subfiles}

\usepackage{amsfonts}
\usepackage{amssymb}
\usepackage{amsmath}

\begin{document}

\section{Vettori}

I vettori sono degli elementi di uno spazio vettoriale, perciò possono essere sommati tra di loro e moltiplicati per dei numeri detti scalari.

\begin{description}
    \item $\vec{v} = (x_1, y_1)$
    \item $\vec{w} = (x_2, y_2)$
\end{description}

\begin{description}
    \item[Somma] $\vec{v} + \vec{w} = (x_1 + x_2, y_1 + y_2)$
    \item[Moltiplicazione] $\lambda \vec{v} = (\lambda x_1, \lambda y_1)$
\end{description}

\subsection{Spazio vettoriale}

Lo spazio vettoriale è una struttura algebrica composta da:

\begin{itemize}
\item Un campo $\mathbb{K}$ di elementi scalari
\item Un insieme di vettori $\mathbb{V}$
\item due operazioni binarie interne (solitamente somma $+$ e prodotto $*$)
    \item $+: \mathbb{V} \times \mathbb{V} \to \mathbb{V}$
    \item $*: \mathbb{K} \times \mathbb{V} \to \mathbb{V}$
\end{itemize}

\noindent
Un spazio vettoriale è tale affinché $(\mathbb{V}, +)$ sia commutativo.

\subsection{Sottospazio vettoriale}

Dati $\mathbb{K}$ campo, $V$ $\mathbb{K}$-spazio vettoriale e $W \subseteq V$, $W$ è un sottospazio vettoriale di $V$ se:

\begin{itemize}
    \item $\vec{0} \in W$
    \item $\forall \ \vec{v}, \vec{w} \in W : \vec{v} + \vec{w} \in W$
    \item $\forall \ \lambda \in \mathbb{K} : \forall \ \vec{v} \in W : \lambda \vec{v} \in W$
\end{itemize}

\subsection{Formula di Grassman}

La formula di Grassman è una relazione tra le dimensioni di due sottospazi di uno stesso spazio vettoriale di dimensione finita.

$$
\dim(U + W) = \dim U + \dim W - \dim(U \wedge W)
$$

\subsection{Somma diretta}

$$
U \oplus W \text{ se } U \cap W = \{ \vec{0} \}
$$

\noindent
$U$ e $W$ sono in somma diretta se e solo se:

$$
B_u \cup B_w = B_{u+w} : \text{ è una base di } U + W
$$

\subsection{Sottospazio affine}

Un sottospazio affine di $V$ è un sottoinsieme nella forma:

$$
\mathbb{A} = \{ \vec{v} + \vec{w} : \vec{w} \in W \}
$$

\end{document}
