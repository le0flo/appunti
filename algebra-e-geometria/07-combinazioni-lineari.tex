\documentclass{article}

\usepackage{amsfonts}
\usepackage{amssymb}
\usepackage{amsmath}

\begin{document}

\section{Combinazioni lineari}

Dato un $\mathbb{V}$ $\mathbb{K}$-spazio vettoriale e degli elementi di $\mathbb{K}$ ($\lambda_i$), si dice combinazione lineare di $n$ vettori $\vec{v}_1, \dots, \vec{v}_n$ una qualunque somma del seguente tipo:

$$
V = \lambda_1 \vec{v}_1 + \dots + \lambda_n \vec{v}_n : \lambda_i \in \mathbb{K}
$$

\noindent
Si può dire che i vettori della combinazione lineare sono:

\begin{description}
    \item[Linearmente dipendenti] se esiste una combinazione lineare a coefficienti non nulli che da come risultato un vettore nullo (i vettori sono paralleli)
    \item[Linearmente indipendenti] se l'unica loro combinazione lineare che da come risultato il vettore nullo, è quella con tutti i coefficienti nulli (i vettori non sono paralleli)
\end{description}

\subsection{Span}

Dati dei vettori $\vec{v}_1, \dots, \vec{v}_n$ di un $\mathbb{V}$ $\mathbb{K}$-spazio vettoriale, lo span è l'insieme di tutte le combinazioni lineari dei vettori con coefficienti $\lambda_i$:

$$
span(\vec{v}_1, \dots, \vec{v}_n) = \{ \lambda_1 \vec{v}_1 + \dots + \lambda_n \vec{v}_n : \lambda_i \in \mathbb{K} \}
$$

\noindent
Lo $span(\vec{v}_1, \dots, \vec{v}_n)$ è un sottospazio vettoriale di $\mathbb{V}$ e si può dire "sottospazio vettoriale generato da $\mathbb{V}$".

\subsection{Base}

I vettori $\vec{v}_1, \dots, \vec{v}_n$ si dicono "sistema di generatori" di $\mathbb{V}$ se ogni vettore di $\mathbb{V}$ si può ottenere come combinazione lineare $\vec{v}_1, \dots, \vec{v}_n$.

\noindent
I vettori $\vec{v}_1, \dots, \vec{v}_n$ sono una base se:

\begin{itemize}
    \item Sono linearmente indipendenti
    \item Sono un sistema di generatori
\end{itemize}

\noindent
Se $\mathbb{V}$ ha una base costituita da $n$ vettori, allora si dice che la dimensione di $\mathbb{V}$ è $n$:

$$
dim \mathbb{V} = n
$$

\noindent
Se conosciamo la dimensione dello spazio vettoriale $\mathbb{V}$ e troviamo un insieme di vettori linearmente indipendenti, allora possiamo dire che quell'insieme di vettori è una base di $\mathbb{V}$.

\end{document}
