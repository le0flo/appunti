\documentclass{article}

\usepackage{amsfonts}
\usepackage{amssymb}
\usepackage{amsmath}

\begin{document}

\section{Probabilità}

\subsection{Definizione classica di probabilità}

Dato un esperimento che presenta più esiti equiprobabili possibili, si definisce probabilità dell'evento $E \subset S$:

$$
P(E) = \frac{n \text{ di esiti favorevoli}}{n \text{ di esisti totali}}
$$

\noindent
Dove il numero di esiti totali deve essere finito.

\subsection{Definizione frequentista}

Ripetuto $N$ volte un esperimento che presenta più esiti possibili, si definisce probabilità dell'evento $E \subset S$:

$$
P(E) = \frac{n \text{ di esperimenti in cui si verifica } E}{N}
$$

\subsection{Principio di enumerazione o fondamentale del calcolo combinatorio}

Si considerino $2$ esperimenti:
il primo presenta $n$ esiti e per ciascuno di questi esiti,
il secondo ne presenta $m$, allora, se le sequenze distinte di esiti danno luogo a risultati diversi,
gli esiti dei due esperimenti saranno complessivamente $n \times m$.

\subsection{Disposizioni semplici}

Si dicono disposizioni semplici di $n$ oggetti di classe $K : (K \leq n)$ tutti gli allineamenti (sequenze ordinate) di $K$ elementi presi dall'insieme di $n$ oggetti.

$$
D_{n,k} = n(n-1)(n-2)\dots(n-k+1)=\frac{n!}{(n-k)!}
$$

\subsection{Disposizioni con ripetizioni}

Si dicono disposizioni con ripetizioni di $n$ oggetti distinti di classe $K$, gli allineamenti che si possono formare prendendo $K$ oggetti non necessariamente diversi dagli altri $n$ dati.

$$
D^R_{n,k} = n^k
$$

\subsection{Permutazioni semplici}

Dati $n$ oggetti distinti, si dicono permutazioni semplici di $n$ elementi gli allineamenti che si possono formare usando tutti gli $n$ oggetti diversi.

$$
P_n = n(n-1)(n-2)\dots1 = n!
$$

\subsection{Permutazioni con ripetizioni}

Dati $n$ oggetti non necessariamente distinti, si definiscono permutazioni con ripetizioni, gli allineamenti che si possono costruire a partire da gli n oggetti dati.

$$
P^R_n = \frac{n!}{K_1!K_2!\dots K_j!}
$$

\noindent
se tra gli $n$ oggetti ce ne sono $K_1$ del primo tipo, $K_2$ del secondo tipo, $\dots$ $K_j$ del j-esimo tipo.

$$
n = \sum^j_{i=1}K_i
$$

\subsection{Combinazioni semplici}

Dati $n$ oggetti distinti, si dicono combinazioni semplici di $n$ oggetti di classe $K (K \leq n)$, gli insiemi che si possono formare a partire dagli $n$ dati:

$$
C_{n,k} = \begin{matrix} n\\ k \end{matrix} = \frac{n!}{(n-k)!k!} = \frac{D_{n,k}}{P_k}
$$

\subsection{Eventi complementari}

Dati $E,F \subset S$, $E^c \subset S$ è l'insieme di tutti gli esiti di $S$ che non stanno in $E$.

\subsubsection{Proprietà}

\begin{itemize}
    \item $E \cup E^c = S$
    \item $E \cap E^c = \emptyset$
    \item $(E \cup F)^c = E^c \cap F^c$
    \item $(E \cap F)^c = E^c \cup F^c$
\end{itemize}

\subsection{Definizione assiomatica di probabilità}

Dato un esperimento a cui è associato uno spazio campione $S$, allora per ogni evento di $S$ è possibile definire un numero $P(E)$, detto probabilità, tale che:

\begin{enumerate}
    \item $P(E) \in [0,1]$
    \item $P(S) = 1$
    \item Dati gli eventi $E_1, E_2, \dots, E_n$ disgiunti: $P(\bigcup^n_{k=1} E_k) = \sum^n_{k=1} P(E_k)$
\end{enumerate}

\subsubsection{Conseguenze immediate}

\begin{enumerate}
    \item Dato uno spazio campione $S$ ed $E \subset S \implies P(E^c) = 1 - P(E)$
    \item $P(\emptyset) = 0$
    \item Dati uno spazio campione $S$ e due eventi $E$ ed $F$ con $E \subset F$, allora $\implies P(E) \leq P(F)$
    \item Dati uno spazio campione $S$ e due eventi $E$ ed $F$, allora $\implies P(E \cup F) = P(E) + P(F) - P(E \cap F)$
\end{enumerate}

\subsection{Probabilità di un evento per uno spazio campio di esiti equiprobabili}

$S = \{e_1, \dots, e_n\}$ per $n \in \mathbb{N} \backslash \{0\}$ e $n < +\infty$\\

\noindent
esiti naturalmente disgiunti: $e_i \cap e_k \emptyset$ se $i \neq k$

$$
S = e_1 \cup \dots \cup e_n = \bigcup^n_{j=1} e_j \implies P(S) = \sum^n_{j=1} P(e_j) = \sum^n_{j=1} p = np \implies p = \frac{1}{n}
$$

\noindent
$E \subset S, E = \{e_1, \dots, e_k\}$

$$
P(E) = P(\bigcup^k_{l=1} e_l) = \sum^k_{l=1} P(e_l) = \sum^k_{l=1} p = kp \implies p = \frac{k}{n}
$$

\noindent
Possiamo vedere che:

$$
P(E) = \frac{k}{n} = \frac{\text{numero di esiti contenuti in } E}{\text{numero di esiti totali in } S} \leftarrow \text{definizione classica di probabilità}
$$

\subsection{Probabilità condizionata}

Dato uno spazio campione $S$ e due eventi $E, F \subset S$ con $P(F) \neq 0$, allora si definisce probabilità condizionata di $E$ condizionata da $F$

$$
P(E | F) = \frac{P(E \cap F)}{P(F)}
$$

\subsection{Due eventi indipendenti}

Dato uno spazio campione $S$ e due eventi $E, F \subset S$, essi si dicono indipendenti se:

$$
P(E \cap F) = P(E)P(F)
$$

\subsubsection{Conseguenze}

Se $P(F) \neq 0$, $P(E|F) = \frac{P(E \cap F)}{P(F)} \implies$ se sono indipendenti $\implies \frac{P(E)P(F)}{P(F)} = P(E)$

\subsubsection{Proprietà}

Dati uno spazio campione $S$ e due eventi indipendenti $E,F \subset S$, allora:

\begin{itemize}
    \item $E$ ed $F^c$ sono indipendenti
    \item $E^c$ ed $F$ sono indipendenti
    \item $E^c$ ed $F^c$ sono indipendenti
\end{itemize}

\subsection{Tre eventi indipendenti}

Dati lo spazio campione $S$ e tre eventi $A,B,C \subset S$, essi si dicono indipendenti se:

$$
\begin{matrix}
P(A \cap B \cap C) = P(A)P(B)P(C) \\
P(A \cap B) = P(A)P(B) \\
P(A \cap C) = P(A)P(C) \\
P(B \cap C) = P(B)P(C)
\end{matrix}
$$

\noindent
Generalizzando, ad $n$ eventi, occorre considerare tutte le intersezioni.

\subsection{Partizione di uno spazio campione}

Dato uno spazio campione $S$, si dice partizione di $S$ una suddivisione in eventi $E_1, \dots, E_n$ tali che:

\begin{itemize}
    \item $\bigcup^n_{k=1} E_k = S$
    \item $E_i \cap E_j = \emptyset$ se $i \neq j$
\end{itemize}

Gli elementi della partizione sono detti ipotesi e si indicano solitamente con la lettera $H$.

\subsection{Formula delle probabilità totali}

Dati uno spazio campione $S$, una sua partizione $\{H_1, \dots, H_n\}$ e un evento $E \subset S$, allora:

$$
P(E) = \sum^n_{k=1} P(E|H_k)P(H_k)
$$

\subsection{Teorema di Bayes}

Dati uno spazio campione $S$, una sua partizione $\{H_1, \dots, H_k\}$ e un evento $E \subset S$ con $P(E) \neq 0$, allora:

$$
P(H_j|E) = \frac{P(E|H_j)P(H_j)}{\sum^n_{k=1} P(E|H_k)P(H_k)} = \frac{P(E|H_j)P(H_j)}{P(E)}
$$

\end{document}
