\documentclass{article}

\usepackage{amsfonts}
\usepackage{amssymb}
\usepackage{amsmath}

\begin{document}

\section{Variabili casuali}

Ad ogni esperimento è possibile associare una variabile casuale a valori reali, i cui valori variano in base all'esito dell'esperimento.

\subsection{Variabili casuali discrete}

$X$ si dice variabile casuale discreta, se $X \in \{x_1, \dots, x_n\}$ con $n \in \mathbb{N} \backslash \{0\}$ per $x_j \in \mathbb{R}$.

\subsection{Funzione di massa di probabilità}

Si associa alla variabile casuale discreta $X$, una funzione:

$$
\begin{matrix}
p &:& \mathbb{R} \to [0,1] \\
p(y) &=& P(X = y)
\end{matrix}
$$

\subsubsection{Funzione di ripartizione o distribuzione di probabilità}

Data la variabile casuale discreta $X$, si associa ad essa la funzione (detta di ripartizione o distribuzione di probabilità):

$$
\begin{matrix}
F &:& \mathbb{R} \to [0,1] \\
F(y) &=& P(X \leq y)
\end{matrix}
$$

\subsubsection{Proprietà}

\begin{itemize}
    \item $p(y) \in [0,1]$
    \item $1 = P(S) = P(X = x_1 \cup \dots \cup X = x_n) = \sum^n_{k=1} P(X = x_k) = \sum^n_{k=1} p(x_k)$
\end{itemize}

\subsection{Variabili casuali continua}

Una variabile casuale $X$ si dice continua se ad essa è associata una funzione (detta di densità di probabilità) $f: \mathbb{R} \to \mathbb{R}^+ \cup \{0\}$ tale che:

$$
P(X \in B) = \int_B f(y)dy
$$

\noindent
$\forall B < \mathbb{R}$

\subsubsection{Funzione di ripartizione o distribuzione di probabilità}

Per una variabile casuale continua $X$:

$$
\begin{matrix}
F(y) &=& P(X \leq y) = \int^y_{-\infty} f(X)dx \\
f(y) &=& \frac{dF(y)}{dy}
\end{matrix}
$$

\subsubsection{Proprietà della funzione di ripartizione di probabilità}

\begin{itemize}
    \item $F(y) \in [0,1] \forall y \in \mathbb{R}$
    \item $\lim_{y \to -\infty} F(y) = P(X \leq -\infty) = 0$
    \item $\lim_{y \to +\infty} F(y) = P(X \leq +\infty) = P(X \in \mathbb{R}) = 1 = P(S)$
    \item $\forall a,b \in \mathbb{R}$ tali che $a < b \implies F(a) \leq F(b)$
\end{itemize}

\subsubsection{Proprietà della funzione di densità di probabilità}

\begin{itemize}
    \item $f(y) \geq 0 \forall y \in \mathbb{R}$
    \item $1 = P(S) = P(X \in \mathbb{R}) = \int^{+\infty}_{-\infty} f(y)dy$
\end{itemize}

\subsubsection{Proprietà}

\begin{itemize}
    \item se $X$ è una variabile casuale continua e $a \in \mathbb{R}$:

    $$
    P(X \leq a) = P(X < a) = \int^a_{-\infty}f(x)dx
    $$

    \item sia $X$ una variabile casuale continua con funzione di ripartizione $F(y)$ e siano $a,b \in \mathbb{R}$ con $a < b$:

    $$
    P(a < X \leq b) = F(b) - F(a)
    $$
\end{itemize}

\subsection{Coppie di variabili casuali discrete}

Dato un esperimento si associano ad esso due variabili casuali $X$ e $Y$ che si dicono discrete se:

$$
\begin{matrix}
X \in \{x_1, \dots, x_n\} \\
Y \in \{y_1, \dots, y_m\} \\
n,m \in \mathbb{N} \backslash \{0\}
\end{matrix}
$$

\subsubsection{Funzione di massa di probabilità congiunta}

$$
\begin{matrix}
p:\mathbb{R}^2 \to [0,1] \\
p(a,b) = P(X=a \cap Y=b) = P(X=a, Y=b) \forall a,b \in \mathbb{R}
\end{matrix}
$$

\noindent
Proprietà:

\begin{itemize}
    \item $p(a,b) \in [0,1] \forall a,b \in \mathbb{R}$
    \item $\sum^n_{i=1} \sum^m_{j=1} p(x_i, y_j) = 1$
\end{itemize}

\subsubsection{Funzioni di massa di probabilità marginali}

$$
\begin{matrix}
p_x(a) = P(X=a) = P(X=a, Y\leq+\infty) = \sum^m_{j=1} p(a, y_j) \text{ con } a \in \mathbb{R} \\
p_y(b) = P(Y=b) = P(X\leq+\infty, Y=b) = \sum^n_{i=1} p(x_i, b) \text{ con } b \in \mathbb{R} \\
\end{matrix}
$$\\

\noindent
esempio, date 12 batterie, di cui 3 cariche, 4 poco cariche e 5 non funzionanti, ne vengono scelte 3 a caso.
$X = \text{numero di batterie cariche estratte}$, $Y = \text{numero di batterie poco cariche estratte}$.
Per $(X,Y)$ determinare $p(a,b)$ e $p_X(a),p_Y(b)$.

\begin{center}
\renewcommand{\arraystretch}{2}
\begin{tabular}{ |c|c|c|c|c|c| }
\hline
$Y \backslash X$ & 0 & 1 & 2 & 3 & $p_Y$ \\
\hline
\hline
0 & $\frac{10}{220}$ & $\frac{30}{220}$ & $\frac{15}{220}$ & $\frac{1}{220}$ & $\frac{56}{220}$ \\
1 & $\frac{40}{220}$ & $\frac{60}{220}$ & $\frac{12}{220}$ & 0 & $\frac{112}{220}$ \\
2 & $\frac{30}{220}$ & $\frac{18}{220}$ & 0 & 0 & $\frac{48}{220}$ \\
3 & $\frac{4}{220}$ & 0 & 0 & 0 & $\frac{4}{220}$ \\
\hline
$p_X$ & $\frac{84}{220}$ & $\frac{108}{220}$ & $\frac{27}{220}$ & $\frac{1}{220}$ & \\
\hline
\end{tabular}
\renewcommand{\arraystretch}{1}
\end{center}

$$
\begin{matrix}
p(0,0) =& P(X=0, Y=0) = \frac{C_{3,0}C_{4,0}C_{5,3}}{C_{12,3}} = \frac{\begin{pmatrix}5\\3\end{pmatrix}}{\begin{pmatrix}12\\3\end{pmatrix}} = \frac{10}{220} \\
p(0,1) =& P(X=0, Y=1) = \frac{C_{3,0}C_{4,1}C_{5,2}}{C_{12,3}} = \frac{4\begin{pmatrix}5\\2\end{pmatrix}}{\begin{pmatrix}12\\3\end{pmatrix}} = \frac{40}{220} \\
\vdots &  \\
p(1,2) =& P(X=1, Y=2) = \frac{C_{3,1}C_{4,2}C_{5,0}}{C_{12,3}} = \frac{40}{220} \\
\end{matrix}
$$

\subsubsection{Funzioni di ripartizione di probabilità congiunta}

$$
\begin{matrix}
F:\mathbb{R}^2 \to [0,1] \\
F(a,b) = P(X \leq a \cap Y \leq b) = P(X \leq a, Y \leq b) \forall a,b \in \mathbb{R}
\end{matrix}
$$

\subsubsection{Funzioni di ripartizione di probabilità marginali}

$$
\begin{matrix}
F_X(a) = P(X \leq a) = P(X \leq a, Y\leq+\infty) = F(a,+\infty) \\
F_Y(b) = P(Y \leq b) = P(X\leq+\infty, Y \leq b) = F(+\infty,b) \\
\end{matrix}
$$

\subsection{Coppie di variabili casuali congiuntamente continue}

Dato un esperimento a cui sono associate le variabili casuali $X$ e $Y$, si dice che esse sono congiuntamente continue se è possibile per esse definire una funzione

$$
f:\mathbb{R}^2 \to \mathbb{R}^+ \cup \{0\}
$$

\noindent
Tale che $\forall D \in \mathbb{R}^2$:

$$
P((X,Y) \in D) = \int_{D}\int f(x,y) dx dy
$$

\noindent
$f$ è detta funzione di densità di probabilità congiunta

\subsubsection{Funzioni di ripartizione di probabilità congiunta}

$$
\begin{matrix}
F:\mathbb{R}^2 \to [0,1] \\
F(a,b) = P(X \leq a \cap Y \leq b)
\end{matrix}
$$

Se $(X,Y)$ sono variabili casuali continue:

$$
F(a,b) = \int^a_{-\infty} (\int^b_{-\infty} f(x,y) dy) dx \implies f(a,b) = \frac{\theta^2 F(a,b)}{\theta a \theta b}
$$

\subsubsection{Funzioni di ripartizione di probabilità marginali}

$$
\begin{matrix}
F_X(a) = P(X \leq a) = P(X \leq a, Y\leq+\infty) = F(a,+\infty) = \int^a_{-\infty} (\int^{+\infty}_{-\infty} f(x,y) dy) dx \\
F_Y(b) = P(Y \leq b) = P(X\leq+\infty, Y \leq b) = F(+\infty,b) = \int^b_{-\infty} (\int^{+\infty}_{-\infty} f(x,y) dx) dy \\
\end{matrix}
$$

\subsubsection{Funzioni di densità di probabilità marginali}

$$
\begin{matrix}
f_X(a) = \int^{+\infty}_{-\infty} f(a,y) dy \\
f_Y(b) = \int^{+\infty}_{-\infty} f(x,b) dx \\
\end{matrix}
$$

\subsection{Coppie di variabili casuali indipendenti}

Data una coppia di variabil casuali, esse si dicono indipendenti se $\forall A,B \subset \mathbb{R}$:

$$
P(X \in A, Y \in B) = P(X \in A)P(Y \in B)
$$

\noindent
Condizione necessaria e sufficiente affinchè una coppia di variabili casuali sia indipendente è che:

\begin{enumerate}
    \item $F(a,b) = F_X(a)F_Y(b) \forall a,b \in \mathbb{R}$

    oppure

    \item se $(X,Y)$ sono una coppia di variabili casuali discrete: $p(a,b) = p_X(a)p_Y(b) \forall a,b \in \mathbb{R}$
    \item se $(X,Y)$ sono congiuntamente continue: $f(a,b) = f_X(a)f_Y(b) \forall a,b \in \mathbb{R}$
\end{enumerate}

\subsection{Valore medio di una variabile casuale}

Data una variabile casuale $X$, si definisce valore atteso o valore medio di $X$, se esiste, il numero:

$$
E[X] =
\begin{cases}
\sum^n_{i=1} x_i p(x_i) & \text{ se $X \in \{x_1, \dots, x_n\}$ (è una variabile casuale discreta)} \\
\int^{+\infty}_{-\infty} x f(x) dx & \text{ se $X$ è una variabile casuale continua}
\end{cases}
$$

\subsubsection{Proprietà}

\begin{itemize}
    \item Data una variabile casuale $X$, sia $Y = h(X)$ una variabile casuale funzione della prima, allora:

    $$
    E[Y] = E[h(X)] =
    \begin{cases}
    \sum^n_{i=1} h(x_i) p(x_i) & \text{ con $X \in \{x_1, \dots, x_n\}$} \\
    \int^{+\infty}_{-\infty} h(x) f(x) dx & \text{ se $X$ è una variabile casuale continua}
    \end{cases}
    $$

    \item Data una variabile casuale $X$ e due numeri reali $\alpha$ e $\beta$:

    $$
    E[\alpha X + \beta] = \alpha E[X] + \beta
    $$

    \item Date due variabil casuali $X$ e $Y$ e definita una variabile casuale $Z = g(X,Y)$, allora:

    $$
    E[Z] = E[g(X,Y)] =
    \begin{cases}
    \sum^n_{i=1} \sum^m_{j=1} g(x_i, y_j) p(x_i, y_j) & \text{ con $X \in \{x_1, \dots, x_n\}$ e $Y \in \{y_1, \dots, y_m\}$} \\
    \int^{+\infty}_{-\infty} \int^{+\infty}_{-\infty} g(x,y) f(x,y) dx dy & \text{ se $(X,Y)$ è una coppia di variabili casuali continue}
    \end{cases}
    $$

    \item Date le variabili casuali $X$ e $Y$, sia $Z = X+Y$, allora:

    $$
    E[Z] = E[X+Y] = E[X] + E[Y] \implies E[\sum^n_{i=1} X_i] = \sum^n_{i=1} E[X_i]
    $$
\end{itemize}

\subsection{Momento $n$-esimo di una variabile casuale}

Data una variabile casuale, si definisce (se esiste) il suo momento $n$-esimo come:

$$
\begin{matrix}
E[X^n] & n \in \mathbb{N}
\end{matrix}
$$

\subsection{Varianza di una variabile casuale}

Data una variabile casuale $X$ di valore medio $E[X]=\mu$, si definisce varianza di $X$ (se esiste):

$$
Var(X) = E[(X-\mu)^2]
$$

\begin{description}
    \item[Se $X$ è una variabile casuale discreta] $Var(x) = \sum^n_{i=1} (x_i-\mu)^2 p(x_i)$
    \item[Se $X$ è una variabile casuale continua] $Var(x) = \int^{+\infty}_{-\infty} (x-\mu)^2 f(x) dx$
\end{description}

\subsubsection{Proprietà}

\begin{itemize}
    \item $Var(X) \geq 0$
    \item $Var(X) = E[X^2] - (E[X])^2$
    \item $Var(\alpha X + \beta) = \alpha^2 Var(X) \text{ se $\alpha, \beta \in \mathbb{R}$}$

    \begin{itemize}
        \item se $\alpha = 0$, $Var(\beta) = 0$
        \item se $\beta = 0$ o $\beta \neq 0$, $Var(\alpha X + \beta) = \alpha^2 Var(X)$
    \end{itemize}

    \item Date due variabili casuali $X$ e $Y$:

    $$
    Cov(X,Y) = E[(X-E[X])(Y-E[Y])]
    $$

    \item Date due variabili casuali $X$ e $Y$:

    $$
    Var(X+Y) = Var(X) + Var(Y) + 2 E[(X-E[X])(Y-E[Y])] = Var(X) + Var(Y) + 2 Cov(X,Y)
    $$
\end{itemize}

\subsection{Covarianza di coppie di variabili casuali}

Date due variabili casuali $X$ e $Y$, di valore medio $E[X]=\mu_X, E[Y]=\mu_Y$, si definisce covarianza di $(X,Y)$ (se esiste):

$$
Cov(X,Y) = E[(X-\mu_X)(Y-\mu_Y)]
$$

\noindent
La covarianza può assumere i seguenti valori, con il rispettivo significato:

\begin{description}
    \item[$Cov(X,Y) < 0$] al crescere (o decrescere) dei valori di una variabile casuale, decrescono (o crescono) quelli dell'altra
    \item[$Cov(X,Y) = 0$] le variabili casuali sono scorrelate
    \item[$Cov(X,Y) > 0$] al crescere (o decrescere) dei valori di una variabile casuale, crescono (o decrescono) quelli dell'altra
\end{description}

\subsubsection{Proprietà}

\begin{itemize}
    \item $Cov(X,Y) = Cov(Y,X)$
    \item $Cov(X,X) = E[(X-\mu_X)^2] = Var(X)$
    \item $Cov(X,Y) = E[XY] - E[X]E[Y]$
    \item Date due variabili casuali indipendenti $X$ e $Y$: $Cov(X,Y) = 0$
    \item $Cov(\alpha X,Y) = Cov(X, \alpha Y) = \alpha Cov(X,Y)$
    \item $Cov(\sum^N_{i=1}X_i, \sum^M_{j=1}Y_j) = \sum^N_{i=1}\sum^M_{j=1} Cov(X_i,Y_i)$
    \item $Var(\sum^N_{i=1}X_i) = \sum^N_{i=1} Var(X_i) + \sum^N_{i=1}\sum^N_{j=1 \text{ con } i \neq j} Cov(X_i, X_j)$
\end{itemize}

\subsection{Coefficiente di correlazione di coppie di variabili casuali}

Date due variabili casuali $X$ e $Y$ di valore atteso $E[X]$ e $E[Y]$, allora si dice coefficiente di correlazione di $X$ e $Y$:

$$
Corr(X,Y) = \frac{Cov(X,Y)}{\sqrt{Var(X)Var(Y)}} \in [-1, 1]
$$

\end{document}
