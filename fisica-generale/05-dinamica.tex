\documentclass{subfiles}

\usepackage{graphicx}
\usepackage{amsfonts}
\usepackage{amssymb}
\usepackage{amsmath}

\graphicspath{ {fisica-generale/assets/} }

\begin{document}

\section{Dinamica}

La dinamica è lo studio delle forze e degli effetti sul moto.
Esistono 4 tipi di forze fondamentali:

\begin{itemize}
    \item Forza elettromagnetica
    \item Forza gravitazionale
    \item Forza "nucleare debole"
    \item Forza "nucleare forte"
\end{itemize}

\noindent
La dinamica classica si basa su tre leggi fondamentali, storicamente formalizzate da Newton.

\subsection{Prima legge}

\subsubsection{Formulazione intuitiva}

Un corpo su cui la risultante delle forze è nulla mantiene inalterato il suo stato di quiete o moto.
Questa enunciazione non è precisa, perchè tratta in modo implicito il sistema di riferimento.

\subsubsection{Formulazione precisa}

Esiste un sistema di riferimento tale che un corpo materiale che sia sufficientemente lontano da tutti gli altri corpi o è in quiete o si muove di moto rettilineo uniforme.
Tale sistema di riferimento è detto inerziale.

\end{document}
